%%
%% This is file `sample-sigconf.tex',
%% generated with the docstrip utility.
%%
%% The original source files were:
%%
%% samples.dtx  (with options: `sigconf')
%%
%% IMPORTANT NOTICE:%%
%% For the copyright see the source file.
%%
%% Any modified versions of this file must be renamed
%% with new filenames distinct from sample-sigconf.tex.
%%
%% For distribution of the original source see the terms
%% for copying and modification in the file samples.dtx.
%%
%% This generated file may be distributed as long as the
%% original source files, as listed above, are part of the
%% same distribution. (The sources need not necessarily be
%% in the same archive or directory.)
%%
%% The first command in your LaTeX source must be the \documentclass command.
\documentclass[sigconf,review, anonymous]{acmart}
\acmConference[ICSE 2024]{46th International Conference on Software Engineering}{April 2024}{Lisbon, Portugal}

\usepackage{code}
\usepackage{graphicx}
\usepackage{balance}
\usepackage{multirow}
\usepackage{multicol}
\usepackage{listings}
\usepackage{booktabs}
\usepackage{subfig}
\usepackage{url}
\usepackage{comment}
\usepackage{xcolor}
\usepackage{xspace}


\definecolor{dkgreen}{rgb}{0,0.5,0}
\definecolor{dkred}{rgb}{0.5,0,0}
\definecolor{gray}{rgb}{0.5,0.5,0.5}

\lstdefinestyle{javastyle} {
language=Java,
basicstyle=\ttfamily\bfseries\footnotesize,
  morekeywords={virtualinvoke},
  keywordstyle=\color{blue},
  ndkeywordstyle=\color{red},
  commentstyle=\color{dkred},
  stringstyle=\color{dkgreen},
  numbers=left,
  breaklines=true,
  numberstyle=\ttfamily\footnotesize\color{gray},
  stepnumber=1,
  numbersep=10pt,
  backgroundcolor=\color{white},
  tabsize=4,
  showspaces=false,
  showstringspaces=false,
  xleftmargin=.23in
}
\lstset{style=javastyle}

%% The "title" command has an optional parameter,
%% allowing the author to define a "short title" to be used in page headers.
\title{Program Mutation is Just Program Transformation: \\ A
  Universal Approach to Mutant Generation}

%%
%% The "author" command and its associated commands are used to define
%% the authors and their affiliations.
%% Of note is the shared affiliation of the first two authors, and the
%% "authornote" and "authornotemark" commands
%% used to denote shared contribution to the research.
%\author{Alex Groce}
%\affiliation{\institution{Northern Arizona University}\country{United States}}


%%
%% By default, the full list of authors will be used in the page
%% headers. Often, this list is too long, and will overlap
%% other information printed in the page headers. This command allows
%% the author to define a more concise list
%% of authors' names for this purpose.

%% Table shortcuts
\newcommand{\mr}[2]{\multirow{#1}{*}{#2}}
\newcommand{\mc}[3]{\multicolumn{#1}{#2}{#3}}
\newcommand{\um}{\texttt{universalmutator}\xspace}
%% comments
\newcommand{\clg}[1]{\textcolor{blue}{#1}}
\newcommand{\adg}[1]{\textcolor{purple}{#1}}
\newcommand{\kj}[1]{\textcolor{olive}{#1}}

%% numbers
\newcommand{\averageprojvariance}{402}
\newcommand{\averagevariance}{604}
\newcommand{\outliertotalfiles}{26}
\newcommand{\outliertestissues}{12}
\newcommand{\outlierumissues}{7}
\newcommand{\outlierunclear}{7}
\newcommand{\allcorr}{0.7479}
\newcommand{\covcorr}{0.2066}
\newcommand{\allrsquared}{0.573}
\newcommand{\allr}{0.757}
\newcommand{\covrsquared}{0.021}
\newcommand{\covr}{0.145}

\begin{document}

\begin{abstract}
While mutation testing has been a topic of academic interest for
decades, it is only recently that ``real-world'' developers, including
industry leaders such as Google and Meta, have adopted mutation
testing.  In this paper we propose a new approach to the development of mutation
testing tools, and in particular the core challenge of
\emph{generating mutants}.  Current practice tends towards two
limited approaches to mutation generation: mutants are either (1)
generated at the bytecode/IR level, and thus neither human readable
nor adaptable to source-level features of languages or projects, or
(2) generated at the source level by language-specific tools that are
hard to write and maintain, and in fact are often abandoned by both
developers and users.  We propose instead that source-level mutation
generation is a special case of \emph{program transformation} in
general, and that adopting this approach allows for a single tool that
can effectively generate source-level mutants for essentially
\emph{any} programming language. Furthermore, by using \emph{parser
  parser combinators} many of the seeming limitations of an
any-language approach can be overcome, without the need to parse
specific languages.  We compare the universal program transformation
approach to mutation to existing tools, and demonstrate the advantages
of using parser-parser combinators to improve on a simple regular-expression
based approach to generation.  
\end{abstract}


\begin{CCSXML}
<ccs2012>
<concept>
<concept_id>10011007.10010940.10010992.10010998.10011001</concept_id>
<concept_desc>Software and its engineering~Dynamic analysis</concept_desc>
<concept_significance>500</concept_significance>
</concept>
<concept>
<concept_id>10011007.10011074.10011099.10011102.10011103</concept_id>
<concept_desc>Software and its engineering~Software testing and debugging</concept_desc>
<concept_significance>500</concept_significance>
</concept>
</ccs2012>
\end{CCSXML}

\ccsdesc[500]{Software and its engineering~Dynamic analysis}
\ccsdesc[500]{Software and its engineering~Software testing and debugging}

\maketitle



\section{Introduction}

Mutation testing, though introduced in the late
1970s~\cite{demillo1978hints,mathur2012foundations,demillo1978hints},
has largely been a topic of academic interest rather than widespread
practice until quite recently.  However, with the adoption of mutation
testing by bellwether software industry companies including
Google~\cite{GoogleMut}, Meta~\cite{BellerFacebookMutation}, and
Amazon~\cite{AmazonMut}, interest in using mutation testing for
real-world projects has grown widely.

Interest in mutation testing has become widespread enough in the open
source community that a popular (more than 300 stars on GitHub)
repository, ``Awesome Mutation Testing,'' lists more than 40 tools,
covering almost twenty languages:
\url{https://github.com/theofidry/awesome-mutation-testing}.  Some of
these tools are the products of academic research, but many are
essentially the hobby projects of developers interested in enabling
mutation testing for their favorite language.

However, the same website demonstrates the existence of a sad history
of mutation testing tools: mutation tools have a tendency to be
developed for a while and then abandoned.   The page listing abandoned
mutation testing tools
(\url{https://github.com/theofidry/awesome-mutation-testing/blob/master/abandoned.md})
lists almost half as many tools as are under active maintenance, and
these tools cover approximately half as many languages.  Each of these
tools required considerable development time and effort, and was
popular enough at some point to be worth noting in the ``Awesome
Mutation Testing'' resource.  Many of the abandoned tools have GitHub
repositories (some archived) with more than 20 stars, and so may well
have had users who at some point faced the reality of no future bug
fixes or changes to address changes to the programming language being
mutated\footnote{While this is a minor concern for languages such as C
  that are relatively stable, it means that a mutation tool may fail
  to parse C++ 2020, and is particularly problematic for somewhat
  fast-moving languages such as Rust.}.  Of course, many open source
projects are abandoned, and academic software has a long history of
being available and supported only so long as a graduate student is
working on a thesis or a particular grant is being funded.  Still,
there appears to be a significant risk in relying on any mutation tools other
than the most widely used and well-supported, such as PIT.

...


In this paper, we argue that developing a large number of mutation
testing tools, each requiring its own development community, workflow
for mutation, and effort to keep up with language changes, is
unnecessary.  Mutation testing, and in particular, the core problem,
\emph{mutation generation} is simply a particular instance of the
problem of \emph{program transformations}.  Moreover,
mutation testing can be treated as an almost entirely \emph{syntactic}
transformation problem; in fact, in their classic introduction to
software testing, Ammann and Offutt refer to mutation testing as
``syntax-based testing'' ~\cite{ammann2016introduction}.  While, e.g., refactoring may need to
``understand'' a program to some extent, mutation can often operate
with essentially no context beyond a single line of code.

We further propose that in fact, for the most part, mutation
generation need not even ``know'' the syntax of a target language, in
terms of a complete grammar.  Instead, mutation testing can operate to
a large extent in a \emph{universal} manner, where programs are
transformed at the level of \emph{patterns of characters}.

The contributions of this paper are:

\begin{enumerate}
  \item The description of a novel approach to mutation generation
    based on generalized description of multi-lingual program transformations
    that allows a single tool to handle mutant generation for
    essentially all programming languages.  This approach also makes
    mutation testing easy to extend for new languages and even allows
    developers to write project-specific rules easily without
    having to modify a mutation tool's implementation.

          \item A proposal to use parser-parser combinators to improve the
      efficiency and expressive power of that approach, and an
      evaluation of the gains thus achieved.

    
      \item A comparison of an implementation of mutation generation
       based on this approach with existing tools for four important
       programming languages.  The single-language tools range from
       approximately 3,800 LOC to nearly 60,000 LOC, and each handle
       one programming language.  Our tool, which provides comparable
       core mutation testing functionality, supports over a dozen
       languages using only about 2200 LOC of Python and less than 500
       lines of rules defining mutation operators.

\end{enumerate}


\section{The Universal Source-Based Approach}

The key insight of this paper is that \emph{most mutation operators
  proposed in the literature do not in fact require parsing of source
  code.}  Consider, for example, one of the most commonly used
mutation operators, replacement of arithmetic operations.  We do not
need to parse a program containing the string {\tt ``x + y''} in order to
replace that string with {\tt ``x - y''}.  Instead, this change is
guaranteed to be effected if we simply search the program, represented
as a string, for all occurrences of  {\tt ``+''}, and produce for each
such occurrence, a mutant where tha character is replaced by {\tt
  ``-''}.  We can do the same for each pair of arithmetic operations,
and thus trivially implement a large fraction of the classic Mothra~\cite{offutt1996experimental}
mutation operators.

The obvious objection to this simplistic approach is that unless we
parse the code to identify arithmetic expressions, some of these
replacements will be invalid.  E.g., in a C++ program we will produce
mutants like {\tt x-+} replacing {\tt x++}.  However, most of these
instances can be avoided by slightly more judicious choice of search
target: instances of {\tt ``+''} where the preceeding or following
character is not another {\tt ``+''}, for instance.

Of course, in some cases this will still leave us with invalid
mutants.  However, rejecting such mutants is easy: the truly invalid
mutants will fail to compile.  Of course, we must pay the price of
checking each mutant to make sure that it does compile, at generation
time.

However, consider what has been gained by paying this cost.  We have
replaced a complex implementation, that requires parsing source code,
with an implementation that only requires string search-and-replace.
One can imagine an implementation of the above approach to arithmetic
operation replacement being implemented in as little as 10-20 lines of
Python code, including a call to a compiler to check for invalid
mutants.  Moreover, defining new mutation operators is not a matter of
understanding a representation of a parsed program, but is essentially
a matter of providing two strings, where the second is a proposed
replacement for the first.  Instead of code for arithmetic operator
replacement, we can imagine the above hypothetical Python program
reading in a file of thus expressed mutation operators, and
implementing an unbounded number of operators in a minimalist fashion.

Of course, simple string replacement is not general enough to handle
many interesting mutations.  Consider the classic case of statement
deletion, one of the most widely used and important mutation
operators~\cite{deng2013empirical}.  What string can we replace with another string in order to
delete a statement in a C program?  We would need an ``operator'' for
each possible C statement.  However, if we allow the use of
\emph{regular expression match and replace pairs} in place of simple
strings, we can easily express this operator, with some precision:

\begin{verbatim}
(^\s*)(\S+[^{}]+.*)\n ==> \1/*\2*/\n
\end{verbatim}

In Python regular expression syntax, this expresses the replacement of
strings with a possibly-empty amount of white space followed by a
character pattern indicating a line of source code, with the same
source code, preserving indentation, turned into a comment.

Using regular expressions to express mutants has many benefits.  While
much more expressive than simple string replacement, regular
expressions are familliar to most developers.  In principle,
development of complex regular expressions for operators can even be
performed by simply providing examples~\cite{bartoli2014automatic}  or
describing the operator in natural
language~\cite{zhong2018generating}.  Implementing a mutation
generation tool for a language by providing the equivalent of our
Python program above with a list of regular expression matches and
replacements is clearly much easier than writing a mutation tool from
scratch, and much easier to maintain and extend. 

\subsection{Hierarchies of Operator Applicability}

Implementing mutation generation by providing a set of regular-expression
based textual transformations for each target language, and using a
single engine to apply rules to source files is certainly less
difficult in terms of development and maintenance effort than building
a tool for each language around a parser for that language.  However,
even this approach ignores one of the primary benefits of a universal,
text-based approach.  Consider the case of replacing instances of {\tt
  ``+''}  with {\tt ``-''}.  This mutation is not tied to any
particular language, but applies to \emph{almost all programming
  languages in use.}  Many of the most widely used mutation operators
in the literature are of this universal nature.  Others are not quite
so widely applicable, but still apply to a large number of specific
languages.  E.g., many programming languages share C's logical
operators, though important exception such as Python and Lisp-family
languages do not.

Implementing a tool for a language $L$ therefore, generally need not
require writing rules for all mutation operators to be applied to $L$
programs, but instead can proceed by process of 1) identifying $L$'s
place in a \emph{hierarchy} of language \emph{famillies}, and then 2)
identifying additional operators needed for $L$ itself.  The first of
these tasks is often trivial, as in practice there are only a few
basic syntactic forms for languages, at the level needed to describe
transformation rules for operators.  In fact, ignoring the second step
will often provide ``good-enough'' mutation testing, in that, for
example, most of the Mothra rules and statement deletion can be
entirely handled without descending to the specific-language level at all.

As an
example, consider mutation of Java code.  Many Java mutants can be
generated by the kind of ``universal'' rules (e.g., arithmetic
operator replacements) considered above.  Other mutation operators
suitable for Java source are provided by considering the set of
``C-like'' languages that use the logical operators and control
constructs common to e.g.,  Java, C, and C++ ({{\tt if}, {\tt while},
  etc.).  Statement deletion at the line level can also be implemented
  by using the common comment notation for such languages.
  Implementing Java mutation, given universal and C-like rules, may
  require no more than a handful of additional rules.

  Figure~\ref{fig:rules} shows examples of universal, C-like, and
  language specific rules for a few languages, taken directly from our
  implementation.  


\subsection{Incremental Mutation}

One obvious concern with any-language source-based mutation is that every time the
program changes, the cost of invalid mutations must be paid again.
However, the locality of mutants and source changes in fact means this
is seldom required.

The {\tt git merge-file} utility takes a base file and two modified
versions of that file, and produces an automatic merge, using the
usual git merge resolution algorithm.    Because a mutant is a (very
small) source change, this means that in practice updating a mutant to
reflect even large changes to a source file is cost-free, and if the
original mutant was valid, the new mutant will usually also be valid.
New mutant generation is only required for new lines of code and
modified lines of code.  To the extent development is incremental,
therefore, mutation generation can also be incremental.

\section{Using Parser Parser Combinators Instead of Regular Expressions}

\section{Implementation}

\section{Experimental Evaluation}

\begin{itemize}

  \item{RQ1:}  How does the use of parser-parser combinators modify 
    the efficiency (in terms of invalid mutants) and effectiveness (in
    terms of mutation score and equivalent mutants) of the universal
    approach to mutation generation?

\item{RQ2:}  How does the univeral approach, using regular expressions
  or parser-combinator combinator rules, compare to existing
  single-language mutation tools, in terms of number of generated
  mutants, mutation scores, equivalent mutants, and other evaluation measures used in the literature?
    
  \end{itemize} 

\subsection{Regular Expressions vs. Parser Parser Combinators}

  
\subsection{Universal Source-Based Mutation vs. Previous Approaches}


\section{Related Work}

Many approaches have been proposed to tackle the \emph{computational} cost of mutation, including weak-mutation, 
meta-mutation, mutation-sampling, and predicting which mutants will be
killed~\cite{offuttMutant1996,
  untch1993mutation,KaufmanFAKAJ2022,zhang2016pmt}.  Approaches to reducing the cost of
mutation analysis were categorized as \textit{do smarter}, \textit{do
faster}, and \textit{do fewer} by Offutt et al.~\cite{offutt2001mutation}.
The \textit{do smarter} approaches include space-time trade-offs, weak
mutation analysis, and parallelization of mutation analysis. The \textit{do
faster} approaches include mutant schema generation, code patching, and
other methods to make mutants run faster. Finally, the
\textit{do fewer} approaches try to reduce the number of mutants examined,
and include selective mutation and mutant sampling.

None of these approaches focus on the cost in \emph{human} time to
develop and maintain mutation testing tools.  In fact, the complexity
and sophistication of some of these approaches imposes a daunting
barrier to those who would develop ``good'' mutation tools for a new
language.  However, arguably the most powerful and generalizable
methods for reducing the cost of mutation, such as random sampling~\cite{GopinathSampleSize,gopinath2017mutation} and
predictive mutation testing~\cite{zhang2016pmt,kim2022predictive}, are independent of the generation of
mutants and so applicable to any language.

Hariri et al. compared C mutation approaches at the source and
compiler IR levels~\cite{CompareSrcBinary} and found that overall
source level mutation was better, producing fewer mutants overall, but
matching the IR approach in the important measures of surface and
minimal mutants and overall mutation score.  Numerous studies compare
Java mutation tools~\cite{MajorPIT,gopinath2017does}, including a
recent article for a more general audience in Communications of the
ACM~\cite{CommACMJavaTool} (perhaps showing the growing interest in
practical mutation testing).  This paper showed that users
considered active maintenance, support for a variety of testing
frameworks, and support for recent Java versions as the most important
features in a Java mutation tool.  The approach proposed in this paper
by its nature tends to promise all three of these key factors without
imposing onerous burdens on maintainers.

Finally, the practical use of mutation testing at Google argues that
support for a variety of languages is critical, and that the approach
we take is more than sufficient.  Google has used its
substantial resources to provide mutation generation for C++, Java,
Python, Javascript, Go, Typescript, and Common
Lisp~\cite{PetrovicMutationGoogle}, all of which we also support.  The
operators Google uses for these
languages (based on the original Mothra operators~\cite{offutt1996experimental}) are a subset of those provided by our implementation, with the caveat
that statement block removal is only supported when using Comby;
regular expression mutation is limited to single-line statement
deletion.  For Lisp-like languages, the difficulty of identifying statements vs. value-returning
function calls makes use of block deletion somewhat impractical, but
still easy to implement.  The majority of the Google Mothra operators
can be implemented in
a few lines as universal rules for all languages; only specializing
logical operators and implementing block/statement deletion even requires
language identification.  Unlike Google's mutation infrastructure, our tool is open source, easily extensible, and not tied to any particular development environment.

\section{Conclusions and Future Work}

\bibliographystyle{ACM-Reference-Format}
\bibliography{bibliography}


\end{document}
