%%
%% This is file `sample-sigconf.tex',
%% generated with the docstrip utility.
%%
%% The original source files were:
%%
%% samples.dtx  (with options: `sigconf')
%%
%% IMPORTANT NOTICE:%%
%% For the copyright see the source file.
%%
%% Any modified versions of this file must be renamed
%% with new filenames distinct from sample-sigconf.tex.
%%
%% For distribution of the original source see the terms
%% for copying and modification in the file samples.dtx.
%%
%% This generated file may be distributed as long as the
%% original source files, as listed above, are part of the
%% same distribution. (The sources need not necessarily be
%% in the same archive or directory.)
%%
%% The first command in your LaTeX source must be the \documentclass command.
\documentclass[sigconf,review, anonymous]{acmart}
\acmConference[ICSE 2024]{46th International Conference on Software Engineering}{April 2024}{Lisbon, Portugal}

\usepackage{code}
\usepackage{graphicx}
\usepackage{balance}
\usepackage{multirow}
\usepackage{multicol}
\usepackage{listings}
\usepackage{booktabs}
\usepackage{subfig}
\usepackage{url}
\usepackage{comment}
\usepackage{xcolor}
\usepackage{xspace}


\definecolor{dkgreen}{rgb}{0,0.5,0}
\definecolor{dkred}{rgb}{0.5,0,0}
\definecolor{gray}{rgb}{0.5,0.5,0.5}

\lstdefinestyle{javastyle} {
language=Java,
basicstyle=\ttfamily\bfseries\footnotesize,
  morekeywords={virtualinvoke},
  keywordstyle=\color{blue},
  ndkeywordstyle=\color{red},
  commentstyle=\color{dkred},
  stringstyle=\color{dkgreen},
  numbers=left,
  breaklines=true,
  numberstyle=\ttfamily\footnotesize\color{gray},
  stepnumber=1,
  numbersep=10pt,
  backgroundcolor=\color{white},
  tabsize=4,
  showspaces=false,
  showstringspaces=false,
  xleftmargin=.23in
}
\lstset{style=javastyle}

%% The "title" command has an optional parameter,
%% allowing the author to define a "short title" to be used in page headers.
\title{Program Mutation is Just Program Transformation: \\ A
  Universal Approach to Mutant Generation}

%%
%% The "author" command and its associated commands are used to define
%% the authors and their affiliations.
%% Of note is the shared affiliation of the first two authors, and the
%% "authornote" and "authornotemark" commands
%% used to denote shared contribution to the research.
%\author{Alex Groce}
%\affiliation{\institution{Northern Arizona University}\country{United States}}


%%
%% By default, the full list of authors will be used in the page
%% headers. Often, this list is too long, and will overlap
%% other information printed in the page headers. This command allows
%% the author to define a more concise list
%% of authors' names for this purpose.

%% Table shortcuts
\newcommand{\mr}[2]{\multirow{#1}{*}{#2}}
\newcommand{\mc}[3]{\multicolumn{#1}{#2}{#3}}
\newcommand{\um}{\texttt{universalmutator}\xspace}
%% comments
\newcommand{\clg}[1]{\textcolor{blue}{#1}}
\newcommand{\adg}[1]{\textcolor{purple}{#1}}
\newcommand{\kj}[1]{\textcolor{olive}{#1}}

%% numbers
\newcommand{\averageprojvariance}{402}
\newcommand{\averagevariance}{604}
\newcommand{\outliertotalfiles}{26}
\newcommand{\outliertestissues}{12}
\newcommand{\outlierumissues}{7}
\newcommand{\outlierunclear}{7}
\newcommand{\allcorr}{0.7479}
\newcommand{\covcorr}{0.2066}
\newcommand{\allrsquared}{0.573}
\newcommand{\allr}{0.757}
\newcommand{\covrsquared}{0.021}
\newcommand{\covr}{0.145}

\begin{document}

\begin{abstract}
While mutation testing has been a topic of academic interest for
decades, it is only recently that ``real-world'' developers, including
industry leaders such as Google and Meta, have adopted mutation
testing.  In this paper we propose a new approach to the development of mutation
testing tools, and in particular the core challenge of
\emph{generating mutants}.  Current practice tends towards two
limited approaches to mutation generation: mutants are either (1)
generated at the bytecode/IR level, and thus neither human readable
nor adaptable to source-level features of languages or projects, or
(2) generated at the source level by language-specific tools that are
hard to write and maintain, and in fact are often abandoned by both
developers and users.  We propose instead that source-level mutation
generation is a special case of \emph{program transformation} in
general, and that adopting this approach allows for a single tool that
can effectively generate source-level mutants for essentially
\emph{any} programming language. Furthermore, by using \emph{parser
  parser combinators} many of the seeming limitations of an
any-language approach can be overcome, without the need to parse
specific languages.  We compare the universal program transformation
approach to mutation to existing tools, and demonstrate the advantages
of using parser-parser combinators to improve on a simple regular-expression
based approach to generation.  The ease of building new tools is
further demonstrated by showing that the implementation of a recently
published mutation tool for Coq can be duplicated using only XX
transformation rules, without parsing the Coq language.
\end{abstract}


\begin{CCSXML}
<ccs2012>
<concept>
<concept_id>10011007.10010940.10010992.10010998.10011001</concept_id>
<concept_desc>Software and its engineering~Dynamic analysis</concept_desc>
<concept_significance>500</concept_significance>
</concept>
<concept>
<concept_id>10011007.10011074.10011099.10011102.10011103</concept_id>
<concept_desc>Software and its engineering~Software testing and debugging</concept_desc>
<concept_significance>500</concept_significance>
</concept>
</ccs2012>
\end{CCSXML}

\ccsdesc[500]{Software and its engineering~Dynamic analysis}
\ccsdesc[500]{Software and its engineering~Software testing and debugging}

\maketitle



\section{Introduction}

\section{Related Work}

Many approaches have been proposed to tackle the \emph{computational} cost of mutation, including weak-mutation, 
meta-mutation, mutation-sampling, and predicting which mutants will be
killed~\cite{offuttMutant1996,
  untch1993mutation,KaufmanFAKAJ2022,zhang2016pmt}.  Approaches to reducing the cost of
mutation analysis were categorized as \textit{do smarter}, \textit{do
faster}, and \textit{do fewer} by Offutt et al.~\cite{offutt2001mutation}.
The \textit{do smarter} approaches include space-time trade-offs, weak
mutation analysis, and parallelization of mutation analysis. The \textit{do
faster} approaches include mutant schema generation, code patching, and
other methods to make mutants run faster. Finally, the
\textit{do fewer} approaches try to reduce the number of mutants examined,
and include selective mutation and mutant sampling.

None of these approaches focus on the cost in \emph{human} time to
develop and maintain mutation testing tools.  In fact, the complexity
and sophistication of some of these approaches imposes a daunting
barrier to those who would develop ``good'' mutation tools for a new
language.  However, arguably the most powerful and generalizable
methods for reducing the cost of mutation, such as random sampling~\cite{GopinathSampleSize,gopinath2017mutation} and
predictive mutation testing~\cite{zhang2016pmt,kim2022predictive}, are independent of the generation of
mutants and so applicable to any language.

Hariri et al. compared C mutation approaches at the source and
compiler IR levels~\cite{CompareSrcBinary} and found that overall
source level mutation was better, producing fewer mutants overall, but
matching the IR approach in the important measures of surface and
minimal mutants and overall mutation score.  Numerous studies compare
Java mutation tools~\cite{MajorPIT,gopinath2017does}, including a
recent article for a more general audience in Communications of the
ACM~\cite{CommACMJavaTool} (perhaps showing the growing interest in
practical mutation testing).  This paper showed that users
considered active maintenance, support for a variety of testing
frameworks, and support for recent Java versions as the most important
features in a Java mutation tool.  The approach proposed in this paper
by its nature tends to promise all three of these key factors without
imposing onerous burdens on maintainers.

Finally, the practical use of mutation testing at Google argues that
support for a variety of languages is critical, and that the approach
we take is more than sufficient.  Google has used its
substantial resources to provide mutation generation for C++, Java,
Python, Javascript, Go, Typescript, and Common
Lisp~\cite{PetrovicMutationGoogle}, all of which we also support.  The
operators Google uses for these
languages (based on the original Mothra operators~\cite{offutt1996experimental}) are a subset of those provided by our implementation, with the caveat
that statement block removal is only supported when using Comby;
regular expression mutation is limited to single-line statement
deletion.  For Lisp-like languages, identifying statements vs. value-returning
function calls makes use of block deletion impractical, but it can be
supported as well.  The majority of these operators are implemented in
a few lines as universal rules for all languages; only specializing
logical operators and implementing block/statement deletion even requires
language identification.  Unlike Google's mutation infrastructure, our tool is open source
, easily extensible, and not tied to any particular development environment.

\section{Conclusion and Implications}

\bibliographystyle{ACM-Reference-Format}
\bibliography{bibliography}


\end{document}
