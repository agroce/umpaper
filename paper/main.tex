%%
%% This is file `sample-sigconf.tex',
%% generated with the docstrip utility.
%%
%% The original source files were:
%%
%% samples.dtx  (with options: `sigconf')
%%
%% IMPORTANT NOTICE:%%
%% For the copyright see the source file.
%%
%% Any modified versions of this file must be renamed
%% with new filenames distinct from sample-sigconf.tex.
%%
%% For distribution of the original source see the terms
%% for copying and modification in the file samples.dtx.
%%
%% This generated file may be distributed as long as the
%% original source files, as listed above, are part of the
%% same distribution. (The sources need not necessarily be
%% in the same archive or directory.)
%%
%% The first command in your LaTeX source must be the \documentclass command.
\documentclass[sigconf,review, anonymous]{acmart}
\acmConference[ICSE 2024]{46th International Conference on Software Engineering}{April 2024}{Lisbon, Portugal}

\usepackage{code}
\usepackage{graphicx}
\usepackage{balance}
\usepackage{multirow}
\usepackage{multicol}
\usepackage{listings}
\usepackage{booktabs}
\usepackage{subfig}
\usepackage{url}
\usepackage{comment}
\usepackage{xcolor}
\usepackage{xspace}

\usepackage{tabularx} % for tables with adjustable column width
\usepackage{makecell} % for formatting table cell contents

\definecolor{dkgreen}{rgb}{0,0.5,0}
\definecolor{dkred}{rgb}{0.5,0,0}
\definecolor{gray}{rgb}{0.5,0.5,0.5}

\lstdefinestyle{javastyle} {
language=Java,
basicstyle=\ttfamily\bfseries\footnotesize,
  morekeywords={virtualinvoke},
  keywordstyle=\color{blue},
  ndkeywordstyle=\color{red},
  commentstyle=\color{dkred},
  stringstyle=\color{dkgreen},
  numbers=left,
  breaklines=true,
  numberstyle=\ttfamily\footnotesize\color{gray},
  stepnumber=1,
  numbersep=10pt,
  backgroundcolor=\color{white},
  tabsize=4,
  showspaces=false,
  showstringspaces=false,
  xleftmargin=.23in
}
\lstset{style=javastyle}

%% The "title" command has an optional parameter,
%% allowing the author to define a "short title" to be used in page headers.
\title{Syntax Is All You Need: A Universal-Language Approach to Mutant Generation}

%%
%% The "author" command and its associated commands are used to define
%% the authors and their affiliations.
%% Of note is the shared affiliation of the first two authors, and the
%% "authornote" and "authornotemark" commands
%% used to denote shared contribution to the research.
%\author{Alex Groce}
%\affiliation{\institution{Northern Arizona University}\country{United States}}


%%
%% By default, the full list of authors will be used in the page
%% headers. Often, this list is too long, and will overlap
%% other information printed in the page headers. This command allows
%% the author to define a more concise list
%% of authors' names for this purpose.

%% Table shortcuts
\newcommand{\mr}[2]{\multirow{#1}{*}{#2}}
\newcommand{\mc}[3]{\multicolumn{#1}{#2}{#3}}
\newcommand{\um}{\texttt{universalmutator}\xspace}
%% comments
\newcommand{\clg}[1]{\textcolor{blue}{#1}}
\newcommand{\adg}[1]{\textcolor{purple}{#1}}
\newcommand{\kj}[1]{\textcolor{olive}{#1}}

%% numbers
\newcommand{\averageprojvariance}{402}
\newcommand{\averagevariance}{604}
\newcommand{\outliertotalfiles}{26}
\newcommand{\outliertestissues}{12}
\newcommand{\outlierumissues}{7}
\newcommand{\outlierunclear}{7}
\newcommand{\allcorr}{0.7479}
\newcommand{\covcorr}{0.2066}
\newcommand{\allrsquared}{0.573}
\newcommand{\allr}{0.757}
\newcommand{\covrsquared}{0.021}
\newcommand{\covr}{0.145}

\begin{document}

\begin{abstract}
While mutation testing has been a topic of academic interest for
decades, it is only recently that ``real-world'' developers, including
industry leaders such as Google and Meta, have adopted mutation
testing.  In this paper we propose a new approach to the development of mutation
testing tools, and in particular the core challenge of
\emph{generating mutants}.  Current practice tends towards two
limited approaches to mutation generation: mutants are either (1)
generated at the bytecode/IR level, and thus neither human readable
nor adaptable to source-level features of languages or projects, or
(2) generated at the source level by language-specific tools that are
hard to write and maintain, and in fact are often abandoned by both
developers and users.  We propose instead that source-level mutation
generation is a special case of \emph{program transformation} in
general, and that adopting this approach allows for a single tool that
can effectively generate source-level mutants for essentially
\emph{any} programming language. Furthermore, by using \emph{parser
  parser combinators} many of the seeming limitations of an
any-language approach can be overcome, without the need to parse
specific languages.  We compare the universal program transformation
approach to mutation to existing tools, and demonstrate the advantages
of using parser parser combinators to improve on a simple regular-expression
based approach to generation.  Finally, we show that our approach
can provide effective mutant generation even for a language for which
it lacks any language-specific operators, and that is not very similar
in syntax to any language it has been applied to previously.
\end{abstract}


\begin{CCSXML}
<ccs2012>
<concept>
<concept_id>10011007.10010940.10010992.10010998.10011001</concept_id>
<concept_desc>Software and its engineering~Dynamic analysis</concept_desc>
<concept_significance>500</concept_significance>
</concept>
<concept>
<concept_id>10011007.10011074.10011099.10011102.10011103</concept_id>
<concept_desc>Software and its engineering~Software testing and debugging</concept_desc>
<concept_significance>500</concept_significance>
</concept>
</ccs2012>
\end{CCSXML}

\ccsdesc[500]{Software and its engineering~Dynamic analysis}
\ccsdesc[500]{Software and its engineering~Software testing and debugging}

\maketitle



\section{Introduction}

Mutation testing, though introduced in the late
1970s~\cite{demillo1978hints,mathur2012foundations,demillo1978hints},
has long been a topic of academic interest.  With the recent adoption of mutation
testing by bellwether software industry companies including
Google~\cite{GoogleMut}, Meta~\cite{BellerFacebookMutation}, and
Amazon~\cite{AmazonMut}, interest in using mutation testing for
real-world projects has grown widely in practice. 
%
Indeed, interest in mutation testing has become widespread enough in the open
source community that a popular (more than 300 stars on GitHub)
repository, ``Awesome Mutation Testing,'' lists more than 40 tools,
covering almost twenty languages:
\url{https://github.com/theofidry/awesome-mutation-testing}.  Some of
these tools are the products of academic research; others are
essentially the hobby projects of developers interested in enabling
mutation testing for their favorite language.

This diversity of tools highlights the growth of widespread interest in mutation
testing.  However, it also demonstrates various ways that bringing mutation testing to
typically polyglot modern projects is cumbersome. Each language (or IR, at
minimum) requires its own underlying mutation and testing framework.
\clg{Alex, can something be said about 
  how many LOC the tools we look at are on average, this early?} 

Newer languages, including ones with considerable interest
in the development community, often lack mutation testing tools.  Even
languages with a long history may long lack a working mutation testing
tool; e.g., it was not until 2014 that Haskell had such a tool~\cite{mucheck}. 
%
Furthermore, the Haskell mutation tool is no longer
supported,\footnote{The last commits or package updates were in 2015, and the
primary author confirms that changes to {\tt haskell-src-ext} since
introduced render it unusable at present.} and it
appears that no successor tool has appeared in the ensuing
years\footnote{There is FitSpec~\cite{FitSpec}, which is maintained,
  but FitSpec does not perform traditional mutation testing or produce
  source code (or bytecode) mutants; it rather
  performs black-box type-aware mutation of function return values.}.
Haskell developers (until now, see below) have no option for
performing true mutation testing.

Indeed, abandonment is a feature of the sad history
of many mutation testing tools, exacerbating the challenges of its uptake.  The page listing abandoned
mutation testing tools
(\url{https://github.com/theofidry/awesome-mutation-testing/blob/master/abandoned.md})
lists almost half as many tools as are under active maintenance. Each tool represents 
considerable development effort, and was
popular enough to be noted in the ``Awesome
Mutation Testing'' resource. Many of the abandoned repositories have more than
20 stars, suggesting inconvenienced users, especially for tools targeting
evolving languages. While this is a minor concern for languages such as C
that are relatively stable, it means that a mutation tool may fail
to parse C++ 2020, and is particularly problematic for 
faster-moving languages like Rust.

%There
%appears to be a significant risk in relying on any mutation tools other
%than the most widely used and well-supported, such as PIT.
A great deal of research effort has been devoted to devising clever
schemes for reducing the computational burden of mutation testing over
the years;\clg{a few references?} in practice, however, the recent adoption of mutation
testing arguably relies more on the availability of vast (cloud) computing
resources than any particular approach from the literature, beyond
heuristic restrictions on the mutants used (e.g., one per line, or
only some siimple operators).   On the other hand, almost no
research has addressed perhaps the most central issue of
mutation testing: in the absence of a working mutation testing
tool for the language in which one is developing software, one cannot
obtain the benefits of mutation testing.  The absence of such tools is
due to the high human cost of developing and maintaining mutation
\clg{OK I don't mind this argument, BUT it kind of seems in conflict with the
  argument above that says there are a lot of tools and mutation testing is
  popular.  I think it might be worth shifting the emphasis of this paragraph to
  be to motivate source-level mutation testing over IR, because that's missing
  so far and would be good to motivate.  Note that it's also an argument made in
  the abstract, which I think is a particularly good summary of the paper.}
testing tools.  Ideally, mutation testing would be available for
\emph{any} newly proposed programming language, as soon as that
language has any community of users at all.  In the present
state-of-the-art, however, it may be years before even rudimentary
tools appear, and they may ``disappear'' shortly there-after.

In this paper, we argue that developing a large number of mutation
testing tools, each requiring its own development community, workflow
for mutation, and effort to keep up with language changes, is fundamentally
unnecessary.  The core problem of mutation testing,
\emph{mutant generation}, is simply a particular instance of the
problem of \emph{program transformation}.  Program transformation
considers the problem of taking a program, $P$, and producing a
program, $P'$ that is the result of applying some function $f$ to $P$
(i.e., $P' = f(P)$).  In mutant generation, we
consider a family of functions $f$ where each $f_{o,loc}$ is an
instantiation of a \emph{mutation operator} $o$  (as classically, if
often informally, defined in the literature) at a particular program
location $loc$ in $P$.  The inclusion of $loc$ in the specification of
$f$ indicates that the problem is a specialization of general
program transformation in that all classic mutation operators we are
aware of are \emph{local}: they only affect a small portion of a
program, genearlly a contiguous string of characters or tokens.

Moreover, as this suggests,
mutation testing can be treated as an almost entirely \emph{syntactic}
transformation problem; in fact, in their classic introduction to
software testing, Ammann and Offutt refer to mutation testing as
``syntax-based testing'' ~\cite{ammann2016introduction}.  While, e.g., refactoring may need to
``understand'' a program to some extent, mutation can often operate
with essentially no context beyond a single line of code.

We further propose that in fact, for the most part, mutant
generation need not even ``know'' the syntax of a target language, in
terms of a complete grammar.  Instead, mutation testing can operate to
a large extent in a \emph{universal} manner, where programs are
transformed at the level of \emph{patterns of characters}.  That is,
each $o$ and $loc$ can be defined without reference to, e.g., the
context-free grammar of a programming language, but only with respect
to a position in $P$ treated as a string, and a transformation on a
string defined in terms of, e.g., regular expressions with capture and
replacement.  Treating mutant generation as a problem of (textual, local)
program transformation provides two major benefits:

\begin{enumerate}
  \item A single tool can provide mutant generation for almost any
    programming language, without needing to parse that language, and
    rules for one language can be re-used for another language.  This
    reduces the development and maintenance effort for mutation
    testing tools by orders of magnitude.  Moreover, it removes the
    need for many   maintenance activities related to parsing code;
    such changes need to be made only when a change to the language
    would introduce new mutation operators or modify existing ones.
    The local nature of operators tends to make the second case very rare.
    \item The existing literature on multi-lingual or language
      agnostic program transformation can be applied to the problem of
      mutant generation, making it easy to express mutation operators
      that operate across many languages, and efficient to generate
      valid mutants even in the absence of language-specific
      development effort.
      \end{enumerate}

The contributions of this paper are therefore:

\begin{enumerate}
  \item The description of a novel approach to mutant generation
    based on generalized description of multi-lingual program transformations
    that allows a single tool to handle mutant generation for
    essentially all programming languages.  This approach also makes
    mutation testing almost trivial to extend to most new languages and even allows
    developers to write project-specific rules easily without
    having to modify a mutation tool's implementation.

          \item A proposal to use parser parser combinators to improve the
      efficiency and expressive power of that approach, and an
      evaluation of the gains thus achieved.

    
      \item A comparison of an implementation of mutant generation
       based on this approach with existing tools for four important
       programming languages.  The single-language tools range from
       approximately 3,800 LOC to nearly 60,000 LOC, and, we emphasize, each handle
       one programming language.  Our tool, which provides comparable
       core mutation testing functionality, supports over a dozen
       languages using only about 2,200 LOC of Python and less than 500
       lines for rules defining mutation operators over a dozen
       specific languages (and which can effectively mutate an
       essentially unlimited number of other languages).

       \item A small case study showing that our approach produces
         useful results, even without the definition of
         language-specific rules, for Haskell, a language arguably
         well outside the language paradigms we considered when
         developing the universal mutation rules.

\end{enumerate}


\section{A Universal Language Syntax-Based Approach}

The key insight of this paper is that \emph{most mutation operators
  proposed in the literature do not in fact require parsing of source
  code.}  Consider, for example, one of the most commonly used
mutation operators, replacement of arithmetic operations.  We do not
need to parse a program containing the string {\tt ``x + y''} in order to
replace that string with {\tt ``x - y''}.  Instead, this change is
guaranteed to be effected if we simply search the program, represented
as a string, for all occurrences of  {\tt ``+''}, and produce for each
such occurrence, a mutant where tha character is replaced by {\tt
  ``-''}.  Many languages express arithmetic as infix operators between
expressions; if expressed syntactically, mutations for them can be expressed
once, trivially implementing a large fraction of the classic Mothra~\cite{offutt1996experimental}
mutation operators.

The obvious objection to this simplistic approach is that unless we
parse the code to identify arithmetic expressions, some of these
replacements will be invalid.  E.g., in a C++ program we will produce
mutants like {\tt x-+} replacing {\tt x++}.  However, most of these
instances can be avoided by slightly more judicious choice of search
target, e.g., instances of {\tt ``+''} where the preceeding or following
character is not another {\tt ``+''}.  This may still in some cases produce
invalid mutants, but rejecting such mutants is easy, since they will fail
to compile.  Of course, we pay the price of
compiling each mutant at generation
time.

On the positive side, this argues that we can replace a
complex implementation requiring parsing and analysis with
withone that only requires string search-and-replace.
Such an implementation on face can be done using much less code. 
% It also allows for simpler definitions of 
% One can imagine an implementation of the above approach to arithmetic
% operation replacement being implemented in as little as 10-20 lines of
% Python code, including a call to a compiler to check for invalid
%mutants.  
Moreover, defining new mutation operators is not a matter of
understanding a parsed program representation, but is essentially
a matter of providing two strings, where the second is a proposed
replacement for the first.  This simplifies operator specification from
implementation of
tree transformations into a minimal list. 
%Instead of code for arithmetic operator
%replacement, we can imagine the above hypothetical Python program
%reading in a file of thus expressed mutation operators, and
%implementing an unbounded number of operators in a minimalist fashion.

Of course, simple string replacement cannot handle
many interesting mutations.  Consider the classic case of statement
deletion, one of the most widely used and important mutation
operators~\cite{deng2013empirical}. This cannot be concisely specified with string pairs. 
%  What string can we replace with another string in order to
% delete a statement in a C program?  We would need an ``operator'' for
% each possible C statement. 
However, if we allow the use of
\emph{regular expression match and replace pairs}, we can express this operator, with some precision:

\begin{verbatim}
(^\s*)(\S+[^{}]+.*)\n ==> \1/*\2*/\n
\end{verbatim}

In Python regular expression syntax, this expresses the replacement of
strings with a possibly-empty amount of white space followed by a
character pattern indicating a line of source code, with the same
source code, preserving indentation, turned into a comment.

\clg{I know we're walking a line on UM anonymity but I think we should cite it
  here, just in the third person.}
Using regular expressions to express mutants has many benefits.  While
much more expressive than simple string replacement, regular
expressions are familliar to most developers.  In principle,
development of complex regular expressions for operators can even be
performed by simply providing examples~\cite{bartoli2014automatic}  or
describing the operator in natural
language~\cite{zhong2018generating}.  
\clg{Just as a note (feel free to read and delete): I tried to cut out our
  hypothetical Python program because I read it and was like OK well then why
  didn't you write it? Or show me if it's so short? You know? And I felt like
  the point could be made without it. I think reference to UM here does the
  trick as well really}
These approaches speak to the potential of string manipulation-based approaches
for mutation testing, without the need for program parsing or analysis. 

% generation tool for a language by providing the equivalent of our
% Python program above with a list of regular expression matches and
% replacements is clearly much easier than writing a mutation tool from
% scratch, and much easier to maintain and extend. 
% CLG: This claim feels strong to me even though I know you think it's true
% because you did it. ;-) 

\subsection{Hierarchies of Operator Applicability}

Regular expressions suggest a mechanism for textual transformation customized to
multiple target languages, which could then share a single engine to apply rules
to source files without the complexity of parsing. 
However, we observe an additional property of mutation operators and mutation
testing that speaks to the potential of a language-agnostic approach: many of the
most widely used mutation operators in the literature are \emph{universal}.  For
example, consider the case of replacing instances of {\tt
  ``+''}  with {\tt ``-''}.  This mutation is not tied to any
particular language, but applies to \emph{almost all programming
  languages in use.}  Others are not quite
so widely applicable, but still apply to a large number of specific
languages.  E.g., many programming languages share C's logical
operators, though important exception such as Python and Lisp-family
languages do not.

Implementing a tool for a language $L$ therefore, generally need not
require writing rules for all mutation operators to be applied to $L$
programs, but instead can proceed by process of 1) identifying $L$'s
place in a \emph{hierarchy} of language \emph{families}, and then 2)
identifying additional operators needed for $L$ itself.  The first of
these tasks is often trivial, as in practice there are only a few
basic syntactic forms for languages, at the level needed to describe
transformation rules for operators.  In fact, ignoring the second step
will often provide ``good-enough'' mutation testing, in that, for
example, most of the Mothra rules and statement deletion can be
entirely handled without descending to the specific-language level at all.

As an
example, consider mutation of Java code.  Many typical Java mutants can be
generated by the kind of ``universal'' rules (e.g., arithmetic
operator replacements) considered above.  Other mutation operators
suitable for Java source are provided by considering the set of
``C-like'' languages that use the logical operators and control
constructs common to e.g.,  Java, C, and C++ ({{\tt if}, {\tt while},
  etc.).  Statement deletion at the line level can also be implemented
  by using the common comment notation for such languages.
  Implementing Java mutation, given universal and C-like rules, may
  require no more than a handful of additional rules.

\begin{table}[hbtp]
\centering
\caption{Rule Examples}
\label{tab:rules}
\resizebox{\columnwidth}{!}{%

  \begin{tabular}{|l|l|l|}
    \hline
    Universal & C-Like & Language-Specific \\
    \hline
{\tt < ==> >} &{\tt  else ==>} & {\tt synchronized ==>} \\ 
{\tt < ==> ==} & {\tt if (\textbackslash(.*\textbackslash)) ==> if (!\textbackslash 1)} &
                                                                 {\tt (\^{}\textbackslash s*)(\textbackslash S+.*\textbackslash n) ==> \textbackslash 1pass}\\
{\tt < ==> <=} & {\tt if (\textbackslash(.*\textbackslash)) ==> if
                 (1==1)} & {\tt any\_of ==> all\_of}\\
\hline

\end{tabular}
}

\end{table}
  
\clg{OK this is confusing and I don't know how to fix it.  Two problems.  (1) we
  haven't actually introduced a tool yet so talking about something from its
  implementation is confusing.  We do have kind of a hypothetical tool, though
  I've been removing specific references to it, because it wasn't clear enough
  that it was supposed to represent our ACTUAL implementation and it was
  confusing me.  If we'd like it to introduce an actual implementation, then we
  might undo my changes there, but make it clear we're introducing an
  implementation and not a thought experiment.  Problem (2) is that a lot of
  this paper (and even the intro/abstract to date) is about using parser
  combinators for this, not regular expressions.  So talking about using regular
  expressions in our not-introduced implementation when alter we'll be talking
  about parser combinators as a contribution is also confusing.  Beyond
  ``concretize our hypothetical tool into introduicng actual tool, other
  solutions I can think of are (1) forward ref a section of an implementation and expand slightly this
  sentence to be like ``taken directly from our implementation of a regular
  expression approach to syntactic mutation (Section FOO)'', or (2) be more abstract
  about it here, like ``Table X shows examples of universal rules...we discuss
  them further and how such rules are used in our implementation in Section
  FOO.''.  Maybe those are the same I dunno.}
Table~\ref{tab:rules} shows examples of universal, C-like, and
language-specific rules for a few languages, taken directly from our
implementation.  In all cases, the form of a rule is: {\tt regexp
  ==> replacement}, using Python regular expression and replacement syntax.

The first column shows universal rules applied to
all languages, here drawn from the large set of transformations for
less-than comparisons.  The second column shows somewhat more
complex rules, for removing elses (which always leaves valid code in
C-like languages, but with the else clause always executing) and for
replacing if conditions with, respectively, a negation and a
constantly true expression.  Finally, the last column shows
language-specific rules, for, respectively, Java, Python (statement
deletion, including a {\tt pass} and respecting indentation level),
and C++.  The last rule was contributed by a user using our tool to
perform mutation testing on the bitcoin core implementation.  For a
C++ program, all of the rules in the first two columns will be
applied, as well as the last rule in the third column.  For a Python
program, however, only the first column of rules and the second rule
in the last column apply, as Python is not sufficiently ``C-like''.


\subsection{Incremental Mutation}

One obvious concern with any-language source-based mutation is that every time the
program changes, the cost of invalid mutations must be paid again.
However, the locality of mutants and source changes in fact means this
is seldom required.

The {\tt git merge-file} utility takes a base file and two modified
versions of that file, and produces an automatic merge, using the
usual git merge resolution algorithm.    Because a mutant is a (very
small) source change, this means that in practice updating a mutant to
reflect even large changes to a source file is cost-free, and if the
original mutant was valid, the new mutant will usually also be valid.
New mutant generation is only required for new lines of code and
modified lines of code.  To the extent development is incremental,
therefore, mutant generation can also be incremental.

\section{Beyond Regular Expressions with Parser Parser Combinators}
% Better to frame this as "Beyond regex with PPC" than "PPC instead of RE" 
% instead of RE" because RE pattern matching can be implemented by PPCs,
% they are just pattern matching functions that can be restricted to lexical
% RE-style matching too, and, e.g., comby embeds PCRE regex matching support,
% as a parser combinator). In this sense, PPCs _really are_ a better/more
% general/more universal matching engine abstraction at the expressivity level
% where we care about mutating source syntax with CFG properties.

\clg{OK.  I think maybe ??? This section works better as a subsection in the
  previous section, which can build to ``beyond...'' Then this text can/should
  start with a TINY bit more transition from the regular expression thing we've
  discussed so far to motivate comby ``instead'' (even though it's not really
  instead.)  This is more a point for discussion than a unilateral decree, but
  it feels like it flows better as a complete story.  Otherwise it's not super
  clear what the point of Section 2 was, if Section 3 is the real thing we're
  proposing? Is it? On that note, if we do this, I think we can/should move
  ``Incremental Mutation'' down to implementation even if we didn't do it, just
  as a point of discussion like ``oh btw we could do it if we wanted it's
  easy.'' Otherwise it kind of disrupts the flow AND hierararies of operators
  leads REALLY REALLY nicely into the parse parser combinator approach.}
\clg{Either way regardless this needs a better starting transition haha.}
We implement a lightweight parser-driven approach that attunes mutation
operators to language-general syntax-aware transformations. For example, our
operators can ensure syntactically well-formed outputs (e.g., transformations
must preserve balanced parentheses or braces). More generally, operators are
sensitive to coarse-grained context-free grammar properties. These operators go
beyond the expressivity of regular expressions (which are typically limited to
lexical transformations) to generate \emph{varied} and \emph{valid}
programs during mutation testing.

Parser parser combinators~\cite{vanTonderPPC} is a recent approach enabling
syntax-transformation for multiple languages, and (we show) an apt choice for
mutation testing operators. This key idea is to replace code constructs
common to many languages (e.g., multi-line code blocks delineated by braces)
with simple declarative patterns. Using parser combinators for coarse syntax
matching avoids the complexity of integrating heterogeneous rewrite tools for
mutation testing (e.g., one for Java, one for C++) and human effort to
implement language-specific mutation operators (e.g., where the user has to
learn to write operators for each language tool). The mechanization and
expressive properties of parser parser combinators is covered in prior
literature~\cite{vanTonderPPC}; in this paper we demonstrate that their application is
uniquely effective for universal source-based mutation testing. In
particular, we
build mutation operators with \texttt{comby},\footnote{https://comby.dev} the tool that implements parser parser
combinators to declaratively match and rewrite source code syntax. 

Comby is especially suited for matching multi-line code blocks and
disambiguating code from comments and strings---constructs that otherwise
confound and complicate regular expression patterns. Comby supports over 40
languages and implements a generic parser for additional languages to recognize
these syntactic constructs. The following is a brief overview of Comby
definitions and syntax for mutation operators implemented in our approach:

\begin{itemize}

\item \texttt{\small:[hole]} syntax matches source code assigned to a variable \emph{hole}. Holes match all characters (including whitespace) lazily up to its suffix (analogous to \texttt{\small.*?} in regex), but only \emph{within} its level of balanced delimiters. For example, \texttt{\small\{:[hole]\}} will match all characters inside balanced braces. By default, parentheses and square brackets are also treated as balanced delimiters, as applicable to most languages. 

\item \texttt{\small ":[x]"} matches the body of a well-quoted string. Unlike \texttt{\small :[x]} (without quotes), the quoted variety implies that a data string may be any value, including strings that contain unbalanced parentheses, like \texttt{\small "item)"}.

\item \texttt{\small:[hole:e]} matches words like \texttt{\small hello} and contiguous well-balanced expression-like syntax, like \texttt{\small print("hello world")} or \texttt{\small (a + b)}. It stops matching at whitespace boundaries, and so does not match a string like \texttt{\small a + b}. It also does not match unbalanced code syntax like \texttt{\small foo)} in typical languages where expressions are expected to be well-balanced.

\item \texttt{\small :[[hole]]} matches alphanumeric characters in source code (similar to \texttt{\small\textbackslash w+} in regex).

\item When a variable \emph{hole} occurs multiple times in a match template, the matched matched values be equal.

\item Non-whitespace characters are matched literally.

\item Contiguous whitespace (e.g., spaces and newline) match contiguous whitespace in the source code. That is, match templates are sensitive to the \emph{presence} of whitespace, but \emph{not} the exact layout of spaces (i.e., the number of spaces may not correspond exactly between match template and source code).

\item Matching is insensitive to comments. Comments are parsed as whitespace when matching non-hole syntax in the match template.

\end{itemize}

Comby provides mechanisms to further define custom match syntax and behavior
beyond these standard patterns.\footnote{https://comby.dev/docs/advanced-usage\#custom-language-definitions} \texttt{Comby} patterns can also embed regular expressions, subsuming their
expressive power to enable regex-based mutation operators intermixed with
context-free syntax properties.\footnote{https://comby.dev/docs/basic-usage\#using-regular-expressions}

\section{Implementation}

We implemented our approach in a mutation testing tool, written in
about 2,200 lines of Python.  The rules defining mutation operators
for a set of languages including C, C++, Python, Java, Lisp, Swift,
Rust, Solidity, Fe, Vyper, JavaScript, and Go, required another 436
lines; this is the total for comby and regexp-based rules, so each
version requires only 218 lines of rule definitions.  The tool makes
use of parser parser combinators via the {\tt comby-python} module.

While many tools provide a GUI front-end, or integrate to a particular
testing library, we chose to focus on an easily configured CLI,
allowing a user to specify the commmands used to build mutants or
run tests.  While less convenient in the case where there is a good
fit between a tool's integrations and a project's environment, we
found this flexibility useful even when limiting considerations to a
single language.  For example, the widely used PIT system is strongly tied to
certain versions of JUnit, and mutating projects using a different
testing infrastructure, or an older JUnit, can be difficult.  Our
approach also is suitable for use in CI and other automated
environments.

The tool attempts to automatically identify which rule sets to use in
mutating a source file, but can also be instructed to run fewer or
more rule sets, and allows users to supply their own custom rule sets.

For the most part we expect users to provide build commands, but do
support automatic compilation and comparison of mutants for Python,
Solidity, Fe, Vyper, Swift, Rust, and Java, including Trivial Compiler
Equivalence~\cite{TCE} checks for redundancy of mutants.  These
default handlers can always be over-ridden if they do not work with a
particular build setup (for Python, the approach almost always works,
but in the case of multi-file projects in other languages, it ususally
requires some manual setup).

The tool also provides a number of utilities, including mechanisms for
incorporating code coverage into mutant selection, integration with a
Python automated testing tool, prioritization of unkilled mutants to
show dis-similar mutants early in a ranking (similar to the FPF
approach to identifying unique bugs found in fuzzing
\cite{10.1145/2491956.2462173}), and tools for pruning a class of
mutants based on some criteria (e.g., removing all mutants of logging
statements from a set of generated mutants).

Our tool has been available as an open source project on GitHub (with
over 100 stars to date) and
available for installation via Python pip for
some time, has received contributions from users not involved in
our project, and has been adopted by at least one flagship project
(over 19K stars) as
its officially recommended mutation tool.  We omit details for
purposes of blinding.

\section{Experimental Evaluation}

\begin{itemize}

  \item{RQ1:}  How does the use of parser parser combinators modify 
    the efficiency (in terms of invalid mutants) and effectiveness (in
    terms of mutation score and equivalent mutants) of the universal
    approach to mutant generation?

\item{RQ2:}  How does the univeral approach, using regular expressions
  or parser parser combinator rules, compare to existing
  single-language mutation tools, in terms of number of generated
  mutants, mutation scores, equivalent mutants, and other evaluation measures used in the literature?

\end{itemize}


Both research questions were addressed by generating and running
mutants for a set of four programming languages: C++, Java, Python,
and Rust.  The criteria for language selection was as follows: C++,
Java, and Python are among the most widely used programming languages
at present.  Furthermore, C++, unlike C, can be notoriously hard to
parse, making development of mutation tools that use an AST and
properly consider mutants of, e.g., code inside a template,
difficult.  Java is the language in which mutation testing has been
studied most extensively in recent literature, and PIT is almost
certainly the most popular and widely used mutation tool.  Python long
suffered from a lack of working and maintained mutation tools, despite
a large number of abandoned efforts, though at present it is served by
multiple maintained, AST-based tools.  Finally, Rust is an ``emerging
language'' with few tools targeting it.

To evaluate both RQ1 and RQ2 we assembled a set of 24 source files,
six from each language.  We first selected, using the GitHub search
API and manual examination, six GitHub
projects for each language meeting a set of criteria:

\begin{enumerate}
\item The project must have a minimum of 1,000 stars.
  \item The percentage of source languages other than the target
    language must be small.
  \item There must be an executable test suite where all tests pass.
    \item The project must be relatively easy to build, without a
      large set of external dependencies.
      \item In addition, for Java, we required use of maven.
      \end{enumerate}

      Files were then selected by picking for each project, the
      largest source file that was less than 20KB in size.  If that
      file was no covered by the test suite, the next closest to 20KB in order
      was chosen.  Finally, if no source file under 20KB existed or
      was covered by tests, the smallest file larger than 20KB covered
      by tests was selected.


We additionally present a small
case study, applying our implementation to a program in a language for
which it has \emph{no} rules, Haskell.  The case study concerns a
fairly trivial example, but was chosen due to being the only program
for which data was available as to the effectiveness of the tool we
compared against (since that tool no longer works).

\subsection{Regular Expressions vs. Parser Parser Combinators}

While the primary advantages of using parser parser combinators as a
basis for mutant generation are in expressiveness and ease of use,
particularly rule readability, a second advantage is a reduction in
the number of invalid mutants generated.

Tables~\ref{tab:table_cpp1}-\ref{tab:table_rust1} compare generated
mutants for ``Regex'' and ``Comby'' modes of our tool.  Invalid
mutants are mutants that the tool produces, but which fail to
compile.  The cost of each such mutant is usually small,
since compilation generally fails early for trivial reasons (e.g.,
{\tt break} or {\tt continue} outside a loop), 
but reducing the number of such mutants directly reduces the cost of
initial mutant generation, and generally indicates a better
``understanding'' of the source being mutated.

% ---------------- C++ table UM Only--------------------------
\begin{table}[hbtp]
\centering
\caption{C++ (Regex vs. Comby)}
\label{tab:table_cpp1}
\resizebox{\columnwidth}{!} & \textbf{LOC} \\ \hline
\multirow{2}{*}{ompl} & \multicolumn{2}{c|}{Regex} & 595 \hspace{8pt} & 1938 \hspace{8pt} & 23.49\% & \multirow{2}{*}{525} \\\cline{2-6}
    & \multicolumn{2}{c|}{Comby} & 564 \hspace{8pt} & 1610 \hspace{8pt} & 25.94\% &  \\ \hline
\multirow{2}{*}{libgraphqlparser} & \multicolumn{2}{c|}{Regex} & 157 \hspace{8pt} & 584 \hspace{8pt} & 21.19\% & \multirow{2}{*}{137}\\\cline{2-6}
    & \multicolumn{2}{c|}{Comby} & 131 \hspace{8pt} & 309 \hspace{8pt} & 29.77\% & \\ \hline
\multirow{2}{*}{rethinkdb\_rebirth} & \multicolumn{2}{c|}{Regex} & 545 \hspace{8pt} & 1701 \hspace{8pt} & 24.27\% & \multirow{2}{*}{471}\\\cline{2-6}
    & \multicolumn{2}{c|}{Comby} & 492 \hspace{8pt} & 1517 \hspace{8pt} & 24.49\% & \\ \hline
    
\multirow{2}{*}{xoreos} & \multicolumn{2}{c|}{Regex} & 2419 \hspace{8pt} & 1457 \hspace{8pt} & 62.41\% & \multirow{2}{*}{732}\\\cline{2-6}
    & \multicolumn{2}{c|}{Comby} & 2280 \hspace{8pt} & 1017 \hspace{8pt} & 69.15\% & \\ \hline
\multirow{2}{*}{libxmljs} & \multicolumn{2}{c|}{Regex} & 881 \hspace{8pt} & 3108 \hspace{8pt} & 22.09\% & \multirow{2}{*}{630}\\\cline{2-6}
    & \multicolumn{2}{c|}{Comby} & 790 \hspace{8pt} & 3218 \hspace{8pt} & 19.71\% & \\ \hline
\multirow{2}{*}{tiny-differentiable-simulator} & \multicolumn{2}{c|}{Regex} & 747 \hspace{8pt} & 535 \hspace{8pt} & 58.27\% & \multirow{2}{*}{195}\\\cline{2-6}
    & \multicolumn{2}{c|}{Comby} & 770 \hspace{8pt} & 444 \hspace{8pt} & 63.43\% & \\ \hline

\multirow{6}{*}{Overall} & \multirow{3}{*}{Regex} & Mean & 890.67 &	1553.83 & 35.28\% & \\\cline{3-6}
    &   &  Median & 671.00 & 1579.00 & 23.88\%  & \\\cline{3-6}
    &   &  Std. Dev & 787.62 & 956.31 & 19.48\% & \\\cline{2-6}

 & \multirow{3}{*}{Comby} & Mean & 837.83 & 1352.50 & 38.75\% & \\\cline{3-6}
    &   &  Median & 667.00 & 1267.00 & 27.86\% & \\\cline{3-6}
    &   &  Std. Dev & 745.78 & 1058.33	& 21.66\% & \\\hline

\end{tabular}
}

\end{table}

  

% ---------------- JAVA table UM Only--------------------------
\begin{table}[hbtp]
\centering
\caption{Java (Regex vs. Comby)}
\label{tab:table_java1}
\resizebox{\columnwidth}{!} & \textbf{LOC}\\ \hline
\multirow{2}{*}{pebble} & \multicolumn{2}{c|}{Regex} & 86 \hspace{8pt} & 928 \hspace{8pt} & 8.48\%  & \multirow{2}{*}{279}\\\cline{2-6}
    & \multicolumn{2}{c|}{Comby} & 85 \hspace{8pt} & 667 \hspace{8pt} & 11.30\% & \\ \hline
\multirow{2}{*}{spring-hateoas} & \multicolumn{2}{c|}{Regex} & 248 \hspace{8pt} & 2231 \hspace{8pt} & 10.00\% & \multirow{2}{*}{397}\\\cline{2-6}
    & \multicolumn{2}{c|}{Comby} & 172 \hspace{8pt} & 1212 \hspace{8pt} & 12.43\% & \\ \hline
\multirow{2}{*}{javacc} & \multicolumn{2}{c|}{Regex} & 542 \hspace{8pt} & 1346 \hspace{8pt} & 28.70\% & \multirow{2}{*}{363}\\\cline{2-6}
    & \multicolumn{2}{c|}{Comby} & 494 \hspace{8pt} & 782 \hspace{8pt} & 38.71\% & \\ \hline
    
\multirow{2}{*}{coffee-gb} & \multicolumn{2}{c|}{Regex} & 224 \hspace{8pt} & 1703 \hspace{8pt} & 11.63\% & \multirow{2}{*}{463}\\\cline{2-6}
    & \multicolumn{2}{c|}{Comby} & 174 \hspace{8pt} & 624 \hspace{8pt} & 21.80\% & \\ \hline
\multirow{2}{*}{egads} & \multicolumn{2}{c|}{Regex} & 624 \hspace{8pt} & 1830 \hspace{8pt} & 25.43\% & \multirow{2}{*}{252}\\\cline{2-6}
    & \multicolumn{2}{c|}{Comby} & 597 \hspace{8pt} & 566 \hspace{8pt} & 51.33\% & \\ \hline
\multirow{2}{*}{apk-parser} & \multicolumn{2}{c|}{Regex} & 351 \hspace{8pt} & 1232 \hspace{8pt} & 22.17\% & \multirow{2}{*}{324}\\\cline{2-6}
    & \multicolumn{2}{c|}{Comby} & 333 \hspace{8pt} & 866 \hspace{8pt} & 27.77\% & \\ \hline

\multirow{6}{*}{Overall} & \multirow{3}{*}{Regex} & Mean & 345.83 & 1545.00  & 17.74\%& \\\cline{3-6}
    &   &  Median & 299.50 & 1524.50 & 16.90\%  & \\\cline{3-6}
    &   &  Std. Dev & 203.86 & 467.98 & 8.73\%  & \\\cline{2-6}

 & \multirow{3}{*}{Comby} & Mean & 309.17 & 786.17 & 27.22\% & \\\cline{3-6}
    &   &  Median & 253.50 & 724.50 & 24.79\%  & \\\cline{3-6}
    &   &  Std. Dev & 202.44 & 235.24  & 15.59\%  & \\\hline


\end{tabular}

}
\end{table}

% ---------------- Python table UM Only --------------------------

\begin{table}[hbtp]
\centering
\caption{Python (Regex vs. Comby)}
\label{tab:table_python1}
\resizebox{\columnwidth}{!} & \textbf{LOC} \\ \hline
\multirow{2}{*}{keract} & \multicolumn{2}{c|}{Regex} & 1655 \hspace{8pt} & 1600 \hspace{8pt} &	729 \hspace{8pt} & 41.53\%  & \multirow{2}{*}{402}\\\cline{2-7}
    & \multicolumn{2}{c|}{Comby} & 1471 \hspace{8pt} & 1063 \hspace{8pt} & 533 \hspace{8pt} & 47.96\% & \\ \hline
\multirow{2}{*}{dtw} & \multicolumn{2}{c|}{Regex} & 973 \hspace{8pt} & 566 \hspace{8pt} & 567 \hspace{8pt} & 46.20\% & \multirow{2}{*}{118}\\\cline{2-7}
    & \multicolumn{2}{c|}{Comby} & 866 \hspace{8pt} & 330 \hspace{8pt} & 639 \hspace{8pt} &	47.19\% & \\ \hline
\multirow{2}{*}{notion-sdk-py} & \multicolumn{2}{c|}{Regex} & 196 \hspace{8pt} & 760 \hspace{8pt} & 82 \hspace{8pt} & 18.89\% & \multirow{2}{*}{189}\\\cline{2-7}
    & \multicolumn{2}{c|}{Comby} & 196 \hspace{8pt} & 701 \hspace{8pt} & 122 \hspace{8pt} & 19.23\% & \\ \hline
    
\multirow{2}{*}{{\makecell{atlassian-python \\ api}}} & \multicolumn{2}{c|}{Regex} & 302 \hspace{8pt}	& 1002 \hspace{8pt} & 207 \hspace{8pt} &	19.99\% & \multirow{2}{*}{236}\\\cline{2-7}
    & \multicolumn{2}{c|}{Comby} & 333 \hspace{8pt} & 635 \hspace{8pt} & 351 \hspace{8pt} & 25.25\% & \\ \hline
\multirow{2}{*}{ESD} & \multicolumn{2}{c|}{Regex} & 1859 \hspace{8pt}	& 2767 \hspace{8pt} & 1642 \hspace{8pt} & 29.46\% & \multirow{2}{*}{698}\\\cline{2-7}
    & \multicolumn{2}{c|}{Comby} & 1712 \hspace{8pt} & 1548 \hspace{8pt} &	614 \hspace{8pt} & 44.18\% & \\ \hline
\multirow{2}{*}{fiber} & \multicolumn{2}{c|}{Regex} & 493 \hspace{8pt} & 909 \hspace{8pt} & 163 \hspace{8pt} & 31.08\% & \multirow{2}{*}{358}\\\cline{2-7}
    & \multicolumn{2}{c|}{Comby} & 458 \hspace{8pt} & 767 \hspace{8pt}	& 241 \hspace{8pt} & 31.24\% & \\ \hline

\multirow{6}{*}{Overall} & \multirow{3}{*}{Regex} & Mean & 913.00 & 1267.33 & 565.00 & 31.19\%  & \\\cline{3-7}
    &   &  Median & 733.00 & 955.50 & 387.00 & 30.27\%  & \\\cline{3-7}
    &   &  Std. Dev & 708.93 & 813.35 & 584.79 & 11.06\%  & \\\cline{2-7}

 & \multirow{3}{*}{Comby} & Mean & 839.33 & 840.67 & 416.67 & 35.84\% & \\\cline{3-7}
    &   &  Median & 662.00 & 734.00 & 442.00 & 37.71\%  & \\\cline{3-7}
    &   &  Std. Dev & 628.79 & 418.99 & 211.62 & 12.28\%  & \\\hline

\end{tabular}

}
\end{table}


% ---------------- Rust table UM Only --------------------------
\begin{table}[hbtp]
\centering
\caption{Rust (Regex vs. Comby)}
\label{tab:table_rust1}
\resizebox{\columnwidth}{!} &  \textbf{LOC} \\ \hline
\multirow{2}{*}{cargo release} & \multicolumn{2}{c|}{Regex} & 851 \hspace{8pt} &	1163 \hspace{8pt} & 42.25\%  & \multirow{2}{*}{346}\\\cline{2-6}
    & \multicolumn{2}{c|}{Comby} & 1690 \hspace{8pt} & 711 \hspace{8pt} & 70.39\% & \\ \hline
\multirow{2}{*}{passerine} & \multicolumn{2}{c|}{Regex} & 213 \hspace{8pt} & 2081 \hspace{8pt} & 9.29\% & \multirow{2}{*}{376}\\\cline{2-6}
    & \multicolumn{2}{c|}{Comby} & 173 \hspace{8pt} & 1518 \hspace{8pt} & 10.23\% & \\ \hline
\multirow{2}{*}{typos} & \multicolumn{2}{c|}{Regex} & 386 \hspace{8pt} & 1534 \hspace{8pt} & 20.10\% & \multirow{2}{*}{463}\\\cline{2-6}
    & \multicolumn{2}{c|}{Comby} & 366 \hspace{8pt} & 1199 \hspace{8pt} &	 23.39\% & \\ \hline
    
\multirow{2}{*}{ord} & \multicolumn{2}{c|}{Regex} & 762 \hspace{8pt}	& 2096 \hspace{8pt} & 26.66\% & \multirow{2}{*}{533}\\\cline{2-6}
    & \multicolumn{2}{c|}{Comby} & 643 \hspace{8pt} & 1669 \hspace{8pt} & 27.81\% & \\ \hline
\multirow{2}{*}{bazuka} & \multicolumn{2}{c|}{Regex} & 451 \hspace{8pt} & 1103 \hspace{8pt} & 29.02\% & \multirow{2}{*}{285}\\\cline{2-6}
    & \multicolumn{2}{c|}{Comby} & 402 \hspace{8pt} & 920 \hspace{8pt} & 30.41\% & \\ \hline
\multirow{2}{*}{strum} & \multicolumn{2}{c|}{Regex} & 51 \hspace{8pt} & 377 \hspace{8pt}	& 11.92\% & \multirow{2}{*}{90}\\\cline{2-6}
    & \multicolumn{2}{c|}{Comby} & 50 \hspace{8pt} & 318 \hspace{8pt}	& 13.51\% & \\ \hline

\multirow{6}{*}{Overall} & \multirow{3}{*}{Regex} & Mean & 452.33 & 1392.33	& 23.21\% & \\\cline{3-6}
    &   &  Median & 418.50 & 1348.50 & 23.38\%  & \\\cline{3-6}
    &   &  Std. Dev & 309.05 & 656.76 & 12.16\%  & \\\cline{2-6}

 & \multirow{3}{*}{Comby} & Mean & 554.00 & 1055.83	& 29.29\% & \\\cline{3-6}
    &   &  Median & 384.00 & 1059.50	& 25.60\%  & \\\cline{3-6}
    &   &  Std. Dev & 592.63 & 508.30 & 21.63\%  & \\\hline


\end{tabular}

}
\end{table}

The mean and median percentage of mutants that are valid is higher for
Comby for all languages.  On average, even for a single modestly sized
file, using parser combinator combinators saves 100-600 failed
compilations.

The number of valid mutants is usually
slightly smaller for Comby, but Comby actually produces a larger mean
(but not median) number of mutants for Rust, due to one project where
Comby produces a much larger number of valid mutants.  As discussed 
below, mutation scores and mutant equivalence rates are similar for
both modes.

For Python, where TCE based on Python bytecode was possible, the
number of redundant (equivalent to the original bytecode, or
equivalent to the bytecode of an already-generated mutant) mutants was
overall similar for both Regex and Comby, though varied widely by file.

In sum, in addition to providing a major advantage in expressiveness
(the ability to easily define complex multi-line mutations that
preserve structure, e.g. going beyond statement deletion to block
deletion), parser parser combinators offer a notable gain in the
efficiency of initial mutant generation, with no apparent loss of overall effectiveness.

\subsection{Universal Source-Based Mutation vs. Previous Approaches}

\begin{table*}[hbtp]
    \centering
    \caption{Overview of Mutation Testing Tools. Our tool has singificantly fewer lines of code than any of the single language tools.}
    \label{tab:mutationtools}
    
    \begin{tabular}{|c|c|c|r|}
    \hline
    \textbf{Language} & \textbf{Tool} & \textbf{URL} & \textbf{LOC}  \\
    \hline
    \multirow{2}{*}{Python}  & Mutmut & https://github.com/boxed/mutmut & 3870  \\\cline{2-4}
        & cosmic-ray & https://github.com/sixty-north/cosmic-ray & 4599 \\ \hline
    \multirow{2}{*}{Java}  & PIT & https://github.com/hcoles/pitest & 59577  \\\cline{2-4}
        & LittleDarwin & https://github.com/aliparsai/LittleDarwin & 22359 \\ \hline
        Rust & Cargo-Mutants & https://github.com/sourcefrog/cargo-mutants & 7020 \\\hline
        C++ & Dextool & https://github.com/joakim-brannstrom/dextool & 38611 \\\hline
        ALL & Our Tool & removed for blinding & 2244 \\\hline
    \end{tabular}
    \end{table*}
    
Table \ref{tab:mutationtools} shows the set of tools compared.  We
attempted to identify well-maintained and commonly used open
source tools for each
language chosen.  The tools cover a number of paradigms; for the most
part we did not choose bytecode based tools, since these often make it
difficult or impossible to mutate a single source file.  PIT, however,
provides this functionality and is perhaps the paradigmatic bytecode
based mutation tool.  Only Dextool, for C++, implements the
\emph{mutant schema}~\cite{untch1993mutation} approach often advocated
in the research literature, where a single metaprogram is produced
and, e.g., environment variabes are used to control which particular
version of a program actually executes.  Using a schema has the major
advantage of dramatically reducing the time required to compile
mutants.  We speculate, however, that the fact that the approach is
somewhat complex to implement for some kinds of mutants, and
furthermore fails if even one invalid mutant is generated, has
prevented most developers of non-academic tools from adopting it.

The most striking thing about this table, and a strong support for
the central argument of this paper, is that our implementation, which
covers \emph{all the languages considered plus a number of additional
languages}, is \emph{the smallest of the tools by a margin of over 1,000 LOC},
and can be extended to handle additional languages with minimal
effort.  Each currently supported language requires about 15 rules on
average, though these must be implemented as both Regex and Comby
rules if both modes are to be used for the language.

The central question is: does using a universal approach produce
notably less effective mutation testing?  Or, to put it another way,
does the substantial effort required to develop and maintain a
mutation tool for a single language pay off in obvious advantages in
mutation testing results?
Tables~\ref{tab:table_cpp2}-\ref{tab:table_rust2} show the results of
mutation testing for the generated mutants for our tool, in both Regex
and Comby modes, and the above-listed tools for each of the four
studied languages.

We do not report the time taken for mutation generation and testing,
since due to only Dextools using schema (and PIT using bytecode mutants), most tools took similar
amounts of time to compile mutants, and mutant execution time was
generally similar, except for tools that produced very few mutants.
We expect that our approach is generally slower for initial mutant
generation, due to a higher number of invalid mutants, but that this
cost is relatively small compared to the time required to compile and
execute valid mutants for almost all projects.

We estimated mutant equivalency when possible by examining a sample of
10 randomly selected unkilled mutants for each file.  For some tools,
human inspection of the individual unkilled mutants was not feasible.
However, we were of course able to estimtate the percent of unkilled
mutants that were unkillable for our two modes for all files, and
these numbers can be put in the perspective of rates reported in the
literature.  Reports of equivalent mutant percentages in the
literature sometimes report the percent of \emph{all} mutants
(including killed mutants, which obviously cannot be equivalent) that
are equivalent.  We follow Schuler and Zeller~\cite{EquivMut} in
reporting the percentage of \emph{unkilled} mutants that were found to
be equivalent.  Schuler and Zeller found about 45\% of unkilled
mutants of Java programs to be equivalent.  Other reports of lower
rates, e.g, 9\%~\cite{offutt1997automatically} or 7\%~\cite{TCE}, report the percent of
\emph{all} mutants, and so translate to much higher rates of
equivalency for unkilled mutants.  We therefore suggest that as a
fraction of unkilled mutants, a rate of 30\% or fewer equivalent
mutants is good, and a rate below 10\% is very good.

% ---------------- C++ table (UM vs Others)--------------------------

\begin{table}[htbp]
\centering
\caption{C++ (Our Implementation vs. Dextool)}
\label{tab:table_cpp2}
\resizebox{\columnwidth}{!}{
\begin{tabular}{|c|c|r|r|r|r|}
\hline
 &  & \textbf{Killed} & \textbf{Not Killed} &  \makecell{\textbf{Mutation} \\ \textbf{Score}} & \textbf{Equivalent \%} \\ \hline

 \multirow{3}{*}{\small ompl} & Regex & 29 & 566 & 0.05 & 30\% \\\cline{2-6}
    & Comby & 41 & 523 & 0.07 & 40\% \\\cline{2-6}
    & Dextool & 23 & 255 & 0.08 & N/A \\ \hline
\multirow{3}{*}{\small libgraphqlparser} & Regex & 142 & 15 & 0.90 & 30\% \\\cline{2-6}
    & Comby & 116 & 15 & 0.89 & 20\% \\ \cline{2-6}
    & Dextool & 50 & 2 & 0.96 & N/A \\ \hline
\multirow{3}{*}{\small rethinkdb\_rebirth} & Regex & 65 & 480 & 0.12 & 10\% \\\cline{2-6}
    & Comby & 49 & 443 & 0.10 & 20\% \\ \cline{2-6}
    & Dextool & 9 & 16 & 0.36 & N/A \\ \hline
    
\multirow{3}{*}{\small xoreos} & Regex & 33 & 2386 & 0.01 & 0\% \\\cline{2-6}
    & Comby & 33 & 2247 & 0.01 & 10\% \\ \cline{2-6}
    & Dextool & 8 & 406 & 0.02 & N/A \\ \hline
\multirow{3}{*}{\small libxmljs} & Regex & 547 & 334 & 0.62 & 10\% \\\cline{2-6}
    & Comby & 507 & 283 & 0.64 & 10\% \\ \cline{2-6}
    & Dextool & 383 & 447 & 0.46 & N/A \\ \hline
\multirow{3}{*}{\small tiny-differentiable-simulator} & Regex & 38 & 709 & 0.05 & 0\% \\\cline{2-6}
    & Comby & 32 & 738 & 0.04 & 0\% \\ \cline{2-6}
    & Dextool & 13 & 48 & 0.21 & N/A\\ \hline
    
\multirow{9}{*}{Overall} & \multirow{3}{*}{Regex} & \multicolumn{2}{c|}{Mean} & 0.29  & 13.33\%\\\cline{3-6}
    &   &  \multicolumn{2}{c|}{Median} & 0.09 & 10.00\% \\\cline{3-6}
    &   &  \multicolumn{2}{c|}{Std. Dev} & 0.38 & 13.66\% \\\cline{2-6}

 & \multirow{3}{*}{Comby} & \multicolumn{2}{c|}{Mean} & 0.29 & 16.67\% \\\cline{3-6}
    &   &  \multicolumn{2}{c|}{Median} & 0.09 & 15.00\% \\\cline{3-6}
    &   &  \multicolumn{2}{c|}{Std. Dev} & 0.37 & 13.66\% \\\cline{2-6}

 & \multirow{3}{*}{Dextool} & \multicolumn{2}{c|}{Mean} & 0.33 & N/A\\\cline{3-6}
    &   &  \multicolumn{2}{c|}{Median} & 0.29 & N/A \\\cline{3-6}
    &   &  \multicolumn{2}{c|}{Std. Dev} & 0.38 & N/A \\ \hline
    
    
\end{tabular}
}
\end{table}

% ---------------- JAVA table (UM vs Others) --------------------------

\begin{table}[htbp]
\centering
\caption{Java (Our Implementation vs. PIT vs. LittleDarwin)}
\label{tab:table_java2}
\resizebox{\columnwidth}{!}{
\begin{tabular}{|c|c|r|r|r|r|}
\hline
 &  & \textbf{Killed} & \textbf{Not Killed} &  \makecell{\textbf{Mutation} \\ \textbf{Score}} & \textbf{Equivalent \%} \\ \hline

 \multirow{4}{*}{\small pebble} & Regex & 70 & 16 & 0.81 & 50\% \\\cline{2-6}
    & Comby & 68 & 17 & 0.80 & 20\% \\\cline{2-6}
    & PIT & 19 & 1 & 0.95 & N/A \\ \cline{2-6}
    & LittleDarwin & 8 & 0 & 1.00 & 0\% \\ \hline
\multirow{4}{*}{\small spring-hateoas} & Regex & 82 & 166 & 0.33 & 0\% \\\cline{2-6}
    & Comby & 65 & 111 & 0.37 & 0\% \\ \cline{2-6}
    & PIT & 82 & 32 & 0.28 & N/A \\ \cline{2-6}
    & LittleDarwin & 15 & 23 & 0.39 & 0\% \\ \hline
\multirow{4}{*}{\small javacc} & Regex & 123 & 419 & 0.23 & 0\% \\\cline{2-6}
    & Comby & 103 & 400 & 0.21 & 0\% \\ \cline{2-6}
    & PIT & 30 & 113 & 0.21 & N/A\\ \cline{2-6}
    & LittleDarwin & 64 & 18 & 0.78 & 0\%\\ \hline
\multirow{4}{*}{\small coffee-gb} & Regex & 48 & 176 & 0.21 & 0\%\\\cline{2-6}
    & Comby & 40 & 134 & 0.23 & 0\% \\ \cline{2-6}
    & PIT & 32 & 4 & 0.89 & N/A \\ \cline{2-6}
    & LittleDarwin & 11 & 18 & 0.38 & 0\%\\ \hline
\multirow{4}{*}{\small egads} & Regex & 442 & 182 & 0.71 & 20\% \\\cline{2-6}
    & Comby & 422 & 200 & 0.71 & 10\% \\ \cline{2-6}
    & PIT & 87 & 20 & 0.81 & N/A \\ \cline{2-6}
    & LittleDarwin & 74 & 11 & 0.87 & 0\% \\ \hline
\multirow{4}{*}{\small apk-parser} & Regex & 26 & 325 & 0.07 & 10\%\\\cline{2-6}
    & Comby & 26 & 312 & 0.08 & 0\%\\ \cline{2-6}
    & PIT & 14 & 75  & 0.16 & N/A\\ \cline{2-6}
    & LittleDarwin & 3 & 30 & 0.09 & 0\%\\ \hline
    
\multirow{12}{*}{Overall} & \multirow{3}{*}{Regex} & \multicolumn{2}{c|}{Mean} & 0.39  & 13.33\%\\\cline{3-6}
    &   &  \multicolumn{2}{c|}{Median} & 0.28 & 5.00\% \\\cline{3-6}
    &   &  \multicolumn{2}{c|}{Std. Dev} & 0.30 & 19.66\% \\\cline{2-6}

 & \multirow{3}{*}{Comby} & \multicolumn{2}{c|}{Mean} & 0.40 & 5.00\%\\\cline{3-6}
    &   &  \multicolumn{2}{c|}{Median} & 0.30 & 0.00\% \\\cline{3-6}
    &   &  \multicolumn{2}{c|}{Std. Dev} & 0.29 & 8.37\%\\\cline{2-6}

 & \multirow{3}{*}{PIT} & \multicolumn{2}{c|}{Mean} & 0.55 & N/A\\\cline{3-6}
    &   &  \multicolumn{2}{c|}{Median} & 0.55 & N/A \\\cline{3-6}
    &   &  \multicolumn{2}{c|}{Std. Dev} & 0.37 & N/A \\\cline{2-6}
    
 & \multirow{3}{*}{LittleDarwin} & \multicolumn{2}{c|}{Mean} & 0.59 & 0.00\%\\\cline{3-6}
    &   &  \multicolumn{2}{c|}{Median} & 0.59 & 0.00\%\\\cline{3-6}
    &   &  \multicolumn{2}{c|}{Std. Dev} & 0.35 & 0.00\%\\ \hline
    
\end{tabular}
}
\end{table}

% ----------------  Python table (UM vs Others) -------------------------------

\begin{table}[htbp]
\centering
\caption{Python (Our Implementation vs. Mutmut vs. CosmicRay)}
\label{tab:table_python2}
\resizebox{\columnwidth}{!}{
\begin{tabular}{|c|c|r|r|r|r|}
\hline
 &  & \textbf{Killed} & \textbf{Not Killed} &  \makecell{\textbf{Mutation} \\ \textbf{Score}} & \textbf{Equivalent \%} \\ \hline

 \multirow{4}{*}{\small keract} & Regex & 463 & 1192 & 0.28 & 0\% \\\cline{2-6}
    & Comby & 383 & 1088 & 0.26 & 10\% \\\cline{2-6}
    & Mutmut & 349 & 133 & 0.72 & 10\% \\ \cline{2-6}
    & CosmicRay & 250 & 671 & 0.27 & 10\% \\ \hline
\multirow{3}{*}{\small dtw} & Regex & 506 & 467 & 0.52 & 10\% \\\cline{2-6}
    & Comby & 475 & 391 & 0.55 & 10\% \\ \cline{2-6}
    & Mutmut & 118 & 152 & 0.44 & 0\% \\ \cline{2-6}
    & CosmicRay & 328 & 385 & 0.46 & 20\% \\ \hline
\multirow{3}{*}{\small notion-sdk-py} & Regex & 112 & 84 & 0.57 & 10\% \\\cline{2-6}
    & Comby & 118 & 78 & 0.60 & 20\% \\ \cline{2-6}
    & Mutmut & 41 & 22 & 0.65 & 0\%\\ \cline{2-6}
    & CosmicRay & 23 & 29 & 0.44 & 0\%\\ \hline
    
\multirow{3}{*}{\makecell{\small atlassian-python \\ api}} & Regex & 193 & 109 & 0.64 & 20\%\\\cline{2-6}
    & Comby & 207 & 126 & 0.62 & 40\% \\ \cline{2-6}
    & Mutmut & 123 & 51 & 0.71 & 10\% \\ \cline{2-6}
    & CosmicRay & 78 & 28 & 0.74 & 10\% \\ \hline
\multirow{3}{*}{\small ESD} & Regex & 132 & 1727 & 0.07 & 30\%\\\cline{2-6}
    & Comby & 121 & 1591 & 0.07 & 10\%\\ \cline{2-6}
    & Mutmut & 57 & 663 & 0.08 & 0\%\\ \cline{2-6}
    & CosmicRay & 63 & 950 & 0.06 & 10\%\\ \hline
\multirow{3}{*}{\small fiber} & Regex & 307 & 186 & 0.62 & 10\%\\\cline{2-6}
    & Comby & 291 & 167 & 0.64 & 20\%\\ \cline{2-6}
    & Mutmut & 127 & 95 & 0.57 & 40\%\\ \cline{2-6}
    & CosmicRay & 259 & 61 & 0.81 & 0\%\\ \hline
    
\multirow{12}{*}{Overall} & \multirow{3}{*}{Regex} & \multicolumn{2}{c|}{Mean} & 0.45  & 13.00\%\\\cline{3-6}
    &   &  \multicolumn{2}{c|}{Median} & 0.55 & 10.00\%\\\cline{3-6}
    &   &  \multicolumn{2}{c|}{Std. Dev} & 0.23 & 10.33\%\\\cline{2-6}

 & \multirow{3}{*}{Comby} & \multicolumn{2}{c|}{Mean} & 0.46 & 18.00\%\\\cline{3-6}
    &   &  \multicolumn{2}{c|}{Median} & 0.58 & 15.00\%\\\cline{3-6}
    &   &  \multicolumn{2}{c|}{Std. Dev} & 0.24 & 11.69\%\\\cline{2-6}

 & \multirow{3}{*}{Mutmut} & \multicolumn{2}{c|}{Mean} & 0.53 & 10.00\%\\\cline{3-6}
    &   &  \multicolumn{2}{c|}{Median} & 0.61 & 5.00\%\\\cline{3-6}
    &   &  \multicolumn{2}{c|}{Std. Dev} & 0.24 & 15.49\%\\\cline{2-6}
    
 & \multirow{3}{*}{CosmicRay} & \multicolumn{2}{c|}{Mean} & 0.46 & 8.33\%\\\cline{3-6}
    &   &  \multicolumn{2}{c|}{Median} & 0.45 & 10.00\%\\\cline{3-6}
    &   &  \multicolumn{2}{c|}{Std. Dev} & 0.28 & 7.53\%\\ \hline
    
\end{tabular}
}
\end{table}


% ---------------- Rust table (UM vs Others)--------------------------

\begin{table}[htbp]
\centering
\caption{Rust (Our Implementation vs. Cargo-Mutants)}
\label{tab:table_rust2}
\resizebox{\columnwidth}{!}{
\begin{tabular}{|c|c|r|r|r|r|}
\hline
 &  & \textbf{Killed} & \textbf{Not Killed} &  \makecell{\textbf{Mutation} \\ \textbf{Score}} & \textbf{Equivalent \%} \\ \hline

 \multirow{3}{*}{\small cargo release} & Regex & 721 & 130 & 0.85 & 0\% \\\cline{2-6}
    & Comby & 1344 & 346 & 0.80 & 0\% \\\cline{2-6}
    & Cargo-Mutants & 2	& 3 & 0.40 & 0\% \\ \hline
\multirow{3}{*}{\small passerine} & Regex & 122 & 91 & 0.57 & 20\% \\\cline{2-6}
    & Comby & 90 & 83 & 0.52 & 50\% \\ \cline{2-6}
    & Cargo-Mutants & 0	& 0 & 0 & 0\% \\ \hline
\multirow{3}{*}{\small typos} & Regex & 295 & 91 & 0.76 & 40\% \\\cline{2-6}
    & Comby & 283 & 83 & 0.77 & 20\%\\ \cline{2-6}
    & Cargo-Mutants & 5 & 1 & 0.83 & 0\%\\ \hline
    
\multirow{3}{*}{\small ord} & Regex & 558 & 204 & 0.73 & 10\% \\\cline{2-6}
    & Comby & 533 & 110 & 0.83 & 20\% \\ \cline{2-6}
    & Cargo-Mutants & 6 & 0 & 1.00 & 0\% \\ \hline
\multirow{3}{*}{\small bazuka} & Regex & 281 & 170 & 0.62 & 40\% \\\cline{2-6}
    & Comby & 291 & 111 & 0.73 & 0\% \\ \cline{2-6}
    & Cargo-Mutants & 0 & 0 & 0 & 0\% \\ \hline
\multirow{3}{*}{\small strum} & Regex & 8 & 43 & 0.16 & 10\%\\\cline{2-6}
    & Comby & 11 & 39 & 0.22 & 40\%\\ \cline{2-6}
    & Cargo-Mutants & 2 & 0 & 1.00 & 0\%\\ \hline
    
\multirow{9}{*}{Overall} & \multirow{3}{*}{Regex} & \multicolumn{2}{c|}{Mean} & 0.62  & 20.00\%\\\cline{3-6}
    &   &  \multicolumn{2}{c|}{Median} & 0.68 & 15.00\%\\\cline{3-6}
    &   &  \multicolumn{2}{c|}{Std. Dev} & 0.24 & 16.73\% \\\cline{2-6}

 & \multirow{3}{*}{Comby} & \multicolumn{2}{c|}{Mean} & 0.64 & 22.00\%\\\cline{3-6}
    &   &  \multicolumn{2}{c|}{Median} & 0.75 & 20.00\% \\\cline{3-6}
    &   &  \multicolumn{2}{c|}{Std. Dev} & 0.25 & 20.41\% \\\cline{2-6}

 & \multirow{3}{*}{Cargo-Mutants} & \multicolumn{2}{c|}{Mean} & 0.54 & 0.00\%\\\cline{3-6}
    &   &  \multicolumn{2}{c|}{Median} & 0.62 & 0.00\% \\\cline{3-6}
    &   &  \multicolumn{2}{c|}{Std. Dev} & 0.47 & 0.00\%\\ \hline
    
    
\end{tabular}
}
\end{table}

Previous studies comparing mutation tools for Java
\cite{MajorPIT,gopinath2017does,CommACMJavaTool} found that results
for individual projects often varied considerably across tools without
indicating a clear advantage for one tool over another.  Our primary
criteria for comparing tools is whether there is an overall pattern of
clearly less-effective mutation for our approach.  Below, we discuss
the question for each language separately.  Each section also
addresses briefly the question of whether Regex and Comby modes
perform differently in terms of mutation score and equivalent mutants,
which is relevant to RQ1.

\subsubsection{C++}

We found only one tool that allowed single-file mutation and worked
reliably for C++.  Dextool generally produced fewer mutants than
either mode of our tool, and for some projects produced an extremely
small number of mutants.  For example, for the {\tt
  tiny-differentiable-simulator} project, Dextool only produced a
total of 61 mutants, and gave a much higher mutation score than our
two modes.  Our tools produced more than 600 additional unkilled
mutants, and none of the inspected mutants were equivalent.
Similarly, Dextool only found 2 unkilled mutants for the {\tt
  libgraphqlparser} project.  We doubt that the set of test weaknesses
in the code is adequately represented by such a small number of
mutants when our tools were able to produce significantly more
unkilled mutants.  Even assuming Dextool's mutants are never
equivalent (highly unlikely) and supposing that a third of our
generated mutants are equivalent, it seems likely that Dextool is
missing important mutants in some cases.  Overall, mean mutation
scores were similar for Dextool and both modes of our tool, while
median mutation scores were much lower for our tools, which produced
many more unkilled mutants for some projects.  For C++, there seems to
be no reason to suspect that the universal approach is producing
sub-par mutation testing.

Comparing Regex and Comby modes, mutation scores are very similar, as
are estimated equivalency rates, though Comby produces a somewhat
larger number and percentage of equivalent mutants.  Both modes
produce relatively low mean percentages of equivalent mutants.

\subsubsection{Java}

For Java, we compare to PIT, arguably the standard for mutation
testing tools in real-world usage (and a tool very frequently used in
the literature) as well as LittleDarwin, a source-based tool.  Our
tool produces lower but comparable mutation scores, and usually
acceptable or excellent equivalency rates.  LittleDarwin produces
\emph{no} equivalent mutants, but is a highly conservative tool, only
implementing a small set of operators, and not including statement
deletion.  Our tools always produced more non-equivalent unkilled
mutants than PIT by a large margin.

Mutation scores for both modes of our tool were again very similar.
For Java, Comby shows a significant advantage in terms of equivalent
mutants, with very low equivalency rates compared to Regex.

\subsubsection{Python}

Our comparison of Python mutation tools showed the largest degree of
similarity between all tools.  Muation scores and equivalency rates
were all, given the large variance expected from previous studies of
Java tools, quite similar.  Again, there is no clear disadvantage to
using our approach, and a large advantage in terms of development and
maintenance ease.  The similarity extends to the two modes of our
tool, though Comby had a somewhat higher equivalency rate than the
other tools, including the Regexp mode of our tool.  We discuss
possible reasons for this below.

\subsubsection{Rust}

Finally, for Rust, we observed that the Cargo-Mutants tool simply does
not generate many mutants.  Mutation scores were broadly similar to
ours, though lower, and none of the generated mutants we examined were
equivalent, but for one project Cargo-Mutants did not produce
\emph{any} mutants even though our tool was able to produce a large
number of non-equivalent unkilled mutants.  Again, behavior of the two
modes of our tool was similar, with both having acceptable equivalency
rates and similar mutation scores.

\subsubsection{Discussion: Summary of Tool Comparisons}

Our tools produce somewhat lower mutation scores than other tools, and
higher (though still generally good) equivalency rates than some
tools, but clearly fall into the same broad similarity of behavior
reported for high-quality mutation tools in the literature comparing
Java mutation tools.   The tools with very low or zero equivalency rates, furthermore,
are very conservative in mutation generation, and often simply don't
produce many mutants.  In fact, inspection of the non-comment source
lines as measured by CLOC shown in Tables~\ref{tab:table_java1} and
\ref{tab:table_java2} and Tables~\ref{tab:table_rust1} and
\ref{tab:table_rust2} shows that for many programs, LittleDarwin and
Cargo-Mutants generated \emph{much less than one mutant per actual
  line code}.  Such a conservative approach is likely to produce few
or no actionable (unkilled and non-equivalent) mutants for a program
with a very good, high-coverage test suite.  But such suites are
arguably the ones where mutation testing offers the most promise, by
allowing the detection of very subtle remaining holes in tests for
high-criticality software.

We speculate that our somewhat lower scores and possibly higher
equivalenc rates derive from the simple fact that we generate more
mutants than most tools.  The ease of defining mutation operators,
plus our interest in generating large numbers of mutants and
prioritizing the most interesting, to make mutation testing useful for
projects with extremely high quality test suites, or even using formal
verification, has encouraged a ``let a thousand flowers bloom''
approach to choosing mutation operators.  For example, our approach
provides many more control-flow mutations than most tools, due to our
observation that test suites sometimes fail to check for loop
execution counts other than one and zero.  Few other tools introduce
{\tt break} and {\tt continue} statements into programs, or transform
one into the other.  These mutants are often equivalent, but also
sometimes expose subtle weaknesses in a test suite, in our
experience.  In general, we take our philosophy of mutation from the
proposal by Gopinath et al.~\cite{Limits,gopinath2017mutation} that
expanding the set of mutation operations and classes of such
operations, in the long run, is more important than using only a small
set of ``good'' operators, in that the end-goal of mutation testing is
to identify weaknesses in test suites, not simply to estimate a single score.

The price paid for this
expansiveness in allowing ways to detect weaknessess in testing is a
larger number of mutants to execute, and a somewhat larger number of
equivalent mutants.  However, for our tool, this observation is not a
fundamental limitation or forced choice.  We propose as future work to
split the operators for each level of the hierarchy into two sets, one
set of core operators similar to (but larger than) those provided by
LittleDarwin or Cargo-Mutants, and a set of ``aggressive'' operators
encompassing our full set of operators, a very simple matter due to
our approach to mutant generation.

\subsection{Haskell Case Study}

\begin{figure}

  \begin{code}
 qsort :: [ Int] -> [ Int ]
 qsort [] = []
 qsort (x: xs ) = qsort l ++ [x] ++ qsort r
     where l = filter (< x) xs
           r = filter ( >= x) xs
\end{code}
\caption{Example from MuCheck Paper}
\label{fig:haskell}

\end{figure}

Figure~\ref{fig:haskell} shows a simple Haskell program, the
motivating example in the paper presenting the MuCheck~\cite{mucheck}
mutation testing tool for Haskell.  As noted in the introduction, to our
knowledge MuCheck is the first (and perhaps only) mutation tool for
Haskell code.  MuCheck has not been updated since 2015.
To see how our approach fares when mutating code in a language for
which 1) it has no specific rules and 2) the assumption of a simple
C-like syntax does not hold, we applied our implementation to this example.

Using regular expressions, and only the universal rules, our tool
generates 10 valid mutants and 34 invalid mutants of the code (22.73\%
valid mutants).  Comby produces an identical 10 valid mutants, but 65
invalid mutants (13.33\% validity).  Both tools produce a large number
of invalid mutants here due to including rules such as adding {\tt
  break} and {\tt continue} that are sufficiently widespread in
applicability to be included in our ``universal'' set but do not apply
to Haskell; we believe that comby's awareness of Haskell syntax in
this case actually gives it \emph{more} opportunities to apply such
inapplicable mutants.

These mutants produce a mutation
score of 0.8 for the two properties defined in the original MuCheck
paper.  In other words, like MuCheck, our tool is able to
determine that the properties provided are insufficient.  In
particular, while they check idempotence and that the result is
sorted, they do not check that the result of {\tt qsort} is a permutaton of the input
list.  The MuCheck paper only provides mutation scores for the two
properties independently.  For idempotence, MuCheck gives a mutation
score of 0.84, and our implementation yields a score of 0.8.  Checking
only sortedness, MuCheck yields a score of 0.61, while our approach
produces an even lower score of 0.4.  Our lower
scores are not due to equivalent mutants: adding a property to check
that the sorted list is a permutation of the original list, all
mutants generated by our approach can be killed.

Of course, our implementation does not provide the small set of Haskell-specific
mutations provided by MuCheck (e.g., type-aware function replacement).  It nonetheless produces clearly useful
mutants, even for a language arguably radically different from those
for which the universal rules were developed.  In short, our approach
likely provides users of a new language working, if sub-optimal, mutation testing,
\emph{even if no effort has been expended to define rules for the
  language at all}.

\section{Threats to Validity}

Our study only examined 24 total source files, from 24 different projects, across four different
languages.  None of these files were very large, though they were of
sizes common to many source files in projects across GitHub, and none
were less than about 100 non-comment lines of code.  More importantly,
there is no simple objective way to determine if a mutation testing
tool is ``as good as'' another tool, in an even approximate sense.
Previous studies of the topic have considered that so long as a tool
generates a significant number of non-equivalent unkilled mutants, and
produces a mutation score not extremely differnt than for other tools,
it is a useful mutation testing tool.  Further, there is no
established acceptable tradeoff between generating more mutants and
accepting a higher cost of compilation and execution for mutants.

\section{Related Work}

Many approaches have been proposed to tackle the \emph{computational} cost of mutation, including weak-mutation, 
meta-mutation, mutation-sampling, and predicting which mutants will be
killed~\cite{offuttMutant1996,
  untch1993mutation,KaufmanFAKAJ2022,zhang2016pmt}.  Approaches to reducing the cost of
mutation analysis were categorized as \textit{do smarter}, \textit{do
faster}, and \textit{do fewer} by Offutt et al.~\cite{offutt2001mutation}.
The \textit{do smarter} approaches include space-time trade-offs, weak
mutation analysis, and parallelization of mutation analysis. The \textit{do
faster} approaches include mutant schema generation, code patching, and
other methods to make mutants run faster. Finally, the
\textit{do fewer} approaches try to reduce the number of mutants examined,
and include selective mutation and mutant sampling.

None of these approaches focus on the cost in \emph{human} time to
develop and maintain mutation testing tools.  In fact, the complexity
and sophistication of some of these approaches imposes a daunting
barrier to those who would develop ``good'' mutation tools for a new
language.  However, arguably the most powerful and generalizable
methods for reducing the cost of mutation, such as random sampling~\cite{GopinathSampleSize,gopinath2017mutation} and
predictive mutation testing~\cite{zhang2016pmt,kim2022predictive}, are independent of the generation of
mutants and so applicable to any language.

Hariri et al. compared C mutation approaches at the source and
compiler IR levels~\cite{CompareSrcBinary} and found that overall
source level mutation was better, producing fewer mutants overall, but
matching the IR approach in the important measures of surface and
minimal mutants and overall mutation score.  Numerous studies compare
Java mutation tools~\cite{MajorPIT,gopinath2017does}, including a
recent article for a more general audience in Communications of the
ACM~\cite{CommACMJavaTool} (perhaps showing the growing interest in
practical mutation testing).  This paper showed that users
considered active maintenance, support for a variety of testing
frameworks, and support for recent Java versions as the most important
features in a Java mutation tool.  The approach proposed in this paper
by its nature tends to promise all three of these key factors without
imposing onerous burdens on maintainers.

Finally, the practical use of mutation testing at Google argues that
support for a variety of languages is critical, and that the approach
we take is more than sufficient.  Google has used its
substantial resources to provide mutant generation for C++, Java,
Python, Javascript, Go, Typescript, and Common
Lisp~\cite{PetrovicMutationGoogle}, all of which we also support.  The
operators Google uses for these
languages (based on the original Mothra operators~\cite{offutt1996experimental}) are a subset of those provided by our implementation, with the caveat
that statement block removal is only supported when using Comby;
regular expression mutation is limited to single-line statement
deletion.  For Lisp-like languages, the difficulty of identifying statements vs. value-returning
function calls makes use of block deletion somewhat impractical, but
still easy to implement.  The majority of the Google Mothra operators
can be implemented in
a few lines as universal rules for all languages; only specializing
logical operators and implementing block/statement deletion even requires
language identification.  Unlike Google's mutation infrastructure, our
tool is open source, easily extensible, and not tied to any particular
development environment.

The approach reported in this paper was briefly presented in part in a
tool paper (citation redacted for blinding purposes) that introduced
the idea of using regular expressions to generate mutants for multiple
languages, and provided an evaluation against PIT for a single toy
Java program.  The present paper provides the full idea and context,
provides empirical data for universal mutant generation on real world
programs over four languages,
and introduces the idea of using parser parser combinators to go
beyond regular expression-based generation.

\section{Conclusions and Future Work}

We conclude that, by using a hierarchical, source-based approach,
either relying on a pure-text regular expression definition of
mutation operators or a richer parser parser combinator transformation
definition, it is possibly to implement effective, easily extensible
mutation testing for essentially \emph{all} commonly used programming
languages, and many non-so-commonly-used programming languages, in
fewer lines of code than any single-language tool we examined.  This
approach fundamentally relies on the insight that program mutation is
just a restricted instance of the general problem of program
transformation, where transformations are highly syntactic (rather
than semantics-aware) and highly localized.

As future work, we plan to give users more control over which mutation
operators are used, with controls common to multiple languages.  We
add that our approach in general seems to be a good fit to modern
large software projects, which are often
multilingual~\cite{Multilingual}: a single mutation tool can be used
for an entire project.

\bibliographystyle{ACM-Reference-Format}
\bibliography{bibliography}


\end{document}
